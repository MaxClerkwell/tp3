\documentclass[a4paper,11pt]{article}
\usepackage[utf8]{inputenc}
\usepackage[T1]{fontenc}
\usepackage{amsmath,amssymb,amsthm}
\usepackage{lmodern}
\usepackage{microtype}
\theoremstyle{definition}
\newtheorem*{remark}{Bemerkung}

\begin{document}

\section*{H3.3 Virialsatz in der Quantenmechanik}

\noindent
Wir betrachten ein eindimensionales Quantensystem mit Hamiltonoperator
\[
H \;=\; T + V \;=\; \frac{p^2}{2m} + V(x)\,,
\]
und eine quadratintegrierbare Eigenfunktion $\psi(x)$ der stationären Schrödinger-Gleichung
\[
H\,\psi(x) \;=\; E\,\psi(x).
\]
Der Erwartungswert eines Operators $A$ im Zustand~$\psi$ sei definiert als
\[
\langle A\rangle_\psi \;=\;
\int_{-\infty}^{\infty}\! \psi^*(x)\,\bigl(A\,\psi(x)\bigr)\,\mathrm{d}x.
\]

\subsection*{(a) Verschwindender Erwartungswert des Kommutators}

\begin{proof}
Sei $f$ ein beliebiger (wohlgewählter) Operator. Wir zeigen:
\[
\bigl\langle [H,f]\bigr\rangle_\psi \;=\; 0.
\]
\begin{align*}
[H,f] &= H\,f \;-\; f\,H,\\
\bigl\langle [H,f]\bigr\rangle_\psi
&= \langle \psi \mid H\,f \mid \psi \rangle
  - \langle \psi \mid f\,H \mid \psi \rangle.
\end{align*}
Da $\psi$ Eigenzustand von $H$ mit Eigenwert $E$ ist, gilt
\[
H\,\psi = E\,\psi
\;\;\Longrightarrow\;\;
\langle \psi \mid H\,f \mid \psi \rangle
= \langle H\,\psi \mid f \mid \psi \rangle
= E\,\langle \psi \mid f \mid \psi \rangle,
\]
bzw.
\[
\langle \psi \mid f\,H \mid \psi \rangle
= \langle \psi \mid f\,(H\,\psi) \rangle
= E\,\langle \psi \mid f \mid \psi \rangle.
\]
Einsetzen liefert
\[
\bigl\langle [H,f]\bigr\rangle_\psi
= E\,\langle f \rangle_\psi - E\,\langle f \rangle_\psi
= 0.
\]
\end{proof}

\subsection*{(b) Herleitung des Virialsatzes für $V(x)=\alpha\,x^N$}

Wir wählen im vorigen Ergebnis $f = p\,x$ und nutzen
\[
p = -\,\mathrm{i}\hbar\,\frac{\mathrm{d}}{\mathrm{d}x},
\qquad
T = \frac{p^2}{2m}.
\]
Dann gilt nach Teil (a)
\[
\bigl\langle [H,\;p\,x]\bigr\rangle_\psi = 0
\quad\Longrightarrow\quad
\bigl\langle [T,\;p\,x]\bigr\rangle_\psi
+
\bigl\langle [V,\;p\,x]\bigr\rangle_\psi
= 0.
\]

\paragraph{1) Kommutator mit der kinetischen Energie:}
\[
[T,\;p\,x]
= \frac{1}{2m}\,[\,p^2,\;p\,x\,]
= \frac{1}{2m}\bigl(p^2\,p\,x - p\,x\,p^2\bigr).
\]
Nutzen wir wiederholt $[A,BC]=[A,B]C + B[A,C]$ sowie
\[
[p,x] = -\,\mathrm{i}\hbar,
\quad
[p,p]=0,
\]
so findet man
\[
[p^2,\;p\,x]
= p\,[p,p\,x] + [p,p\,x]\,p
= -2\,\mathrm{i}\hbar\,p^2
\;\;\Longrightarrow\;\;
[T,\;p\,x]
= -\,\mathrm{i}\hbar\,\frac{p^2}{m}
= -2\,\mathrm{i}\hbar\,T.
\]
Daher
\[
\bigl\langle [T,\;p\,x]\bigr\rangle_\psi
= -2\,\mathrm{i}\hbar\,\langle T\rangle_\psi.
\]

\paragraph{2) Kommutator mit dem Potential:}
Da $V=V(x)$ nur von $x$ abhängt,
\[
[V,\;p\,x]
= V\,p\,x - p\,x\,V
= [V,p]\,x
\quad\text{und}\quad
[V,p] = -\,\mathrm{i}\hbar\,V'(x),
\]
ergibt sich
\[
[V,\;p\,x]
= -\,\mathrm{i}\hbar\,x\,V'(x),
\]
also
\[
\bigl\langle [V,\;p\,x]\bigr\rangle_\psi
= -\,\mathrm{i}\hbar\,\bigl\langle x\,V'(x)\bigr\rangle_\psi.
\]

\paragraph{3) Zusammenfügen und Vereinfachen:}
\[
0
= \bigl\langle [T,\;p\,x]\bigr\rangle_\psi
+ \bigl\langle [V,\;p\,x]\bigr\rangle_\psi
= -2\,\mathrm{i}\hbar\,\langle T\rangle_\psi
  - \mathrm{i}\hbar\,\langle x\,V'(x)\rangle_\psi.
\]
Division durch $-\mathrm{i}\hbar$ liefert
\[
2\,\langle T\rangle_\psi
\;-\;
\bigl\langle x\,V'(x)\bigr\rangle_\psi
= 0.
\]
Für das Potential $V(x)=\alpha\,x^N$ ist
\[
V'(x) = \alpha\,N\,x^{N-1},
\qquad
x\,V'(x) = N\,\alpha\,x^N = N\,V(x),
\]
und somit der \emph{Virialsatz}
\[
\boxed{%
2\,\langle T\rangle_\psi
\;-\;
N\,\langle V\rangle_\psi
=0.
}
\]

\begin{remark}
Der Virialsatz liefert eine wichtige Beziehung zwischen kinetischer und potentieller Energie
für Potenziale mit Potenzgesetz $x^N$.
\end{remark}

\end{document}
