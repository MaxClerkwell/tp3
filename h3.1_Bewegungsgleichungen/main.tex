% main.tex – Ausführliche Herleitung der Bewegungsgleichung für ⟨x²⟩
\documentclass[a4paper,12pt]{article}
\usepackage[utf8]{inputenc}
\usepackage[T1]{fontenc}
\usepackage{amsmath,amssymb}
\usepackage{physics}
\usepackage{geometry}
\geometry{margin=2.5cm}

\begin{document}

\title{Ausführliche Herleitung der Bewegungsgleichung für $\langle x^2 \rangle$}
\author{}
\date{}
\maketitle

\section*{Einleitung}
In der Quantenmechanik beschreibt der Erwartungswert eines Operators die physikalisch gemessenen Mittelwerte. Für die Ortsquadratskomponente $x^2$ interessiert uns, wie sich dieser Erwartungswert im Laufe der Zeit entwickelt. Mithilfe des Ehrenfest-Theorems lässt sich eine Bewegungsgleichung für $\langle x^2\rangle$ ableiten, die ein quantenmechanisches Analogon zur klassischen Bewegungsgleichung darstellt.

\section{Theoretische Grundlagen}
\subsection{Operatoren und Erwartungswerte}
Sei $A$ ein Operator im Hilbertraum eines quantenmechanischen Systems mit Zustandsvektor $\ket{\psi(t)}$. Dann definiert sich der Erwartungswert von $A$ durch

$$
  \langle A \rangle(t) = \bra{\psi(t)} A \ket{\psi(t)}.
$$

Der Zeitverlauf von $\ket{\psi(t)}$ wird durch die Schrödinger-Gleichung bestimmt:

$$
  i\hbar \frac{\partial}{\partial t} \ket{\psi(t)} = H \ket{\psi(t)},
$$

wobei \$H\$ der Hamiltonoperator ist.

\subsection{Ehrenfest-Theorem}
Für einen zeitunabhängigen Operator $A$ gilt gemäß dem Ehrenfest-Theorem:

$$
  \frac{d}{dt} \langle A \rangle = \frac{i}{\hbar} \langle [H, A] \rangle.
$$

\subsection{Hamiltonoperator des Teilchens}
Wir betrachten ein Teilchen der Masse $m$ im eindimensionalen Potential $V(x)$. Der Hamiltonoperator lautet:

$$
  H = \frac{p_x^2}{2m} + V(x),
$$

wobei $p_x$ der Impulsoperator in $x$-Richtung ist mit der kanonischen Vertauschungsrelation $[x, p_x] = i\hbar$.

\section{Herleitung für $A = x^2$}
Da $x^2$ keine explizite Zeitabhängigkeit besitzt, folgt aus dem Ehrenfest-Theorem:

$$
  \frac{d}{dt} \langle x^2 \rangle = \frac{i}{\hbar} \langle [H, x^2] \rangle.
$$

\subsection{Kommutator $[H, x^2]$}
Weil $[V(x), x^2] = 0$ (beide sind Funktionen von $x$), ergibt sich:

$$
  [H, x^2] = \frac{1}{2m} [p_x^2, x^2].
$$

Mit der Produktregel für Kommutatoren $[B^2, C] = B[B,C] + [B,C]B$ und

$$
  [p_x, x^2] = p_x x^2 - x^2 p_x = -2i\hbar \, x,
$$

bekommt man:

$$
  [p_x^2, x^2] = p_x [p_x, x^2] + [p_x, x^2] p_x = -2i\hbar (p_x x + x p_x).
$$

\subsection{Endgültige Bewegungsgleichung}
Einsetzen in die Zeitableitung liefert:

$$
  \frac{d}{dt} \langle x^2 \rangle
  = \frac{i}{\hbar} \cdot \frac{1}{2m} \langle -2i\hbar (p_x x + x p_x) \rangle
  = \frac{\langle x p_x \rangle + \langle p_x x \rangle}{m}.
$$

\section*{Schlussfolgerung}
Es wurde gezeigt, dass der Erwartungswert $\langle x^2 \rangle$ die Bewegungsgleichung

$$
  \frac{d}{dt} \langle x^2 \rangle = \frac{\langle x p_x \rangle + \langle p_x x \rangle}{m}
$$

erfüllt. Diese Beziehung ist eine direkte Konsequenz des Ehrenfest-Theorems und verbindet quantenmechanische Operatoren mit klassischen physikalischen Größen.

\end{document}
