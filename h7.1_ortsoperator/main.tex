\documentclass{scrartcl}
\usepackage{amsmath,amssymb}
\usepackage[ngerman]{babel}

\begin{document}

\section*{Lösung zu H7.1: Darstellung des Ortsoperators im unendlich tiefen Potentialtopf}

Wir betrachten ein Teilchen im ein-dimensionalen unendlich tiefen Potentialtopf der Breite $a$, d.\,h.
\[
V(x) \;=\;
\begin{cases}
0, & 0 < x < a,\\
\infty, & \text{sonst}.
\end{cases}
\]
Die normierten Energieeigenfunktionen lauten (für $n=1,2,\dots$):
\[
\Psi_n(x) \;=\; \sqrt{\frac{2}{a}} \,\sin\!\biggl(\frac{n\pi x}{a}\biggr).
\]
Ziel ist es, die Matrixelemente des Ortsoperators $\hat{x}$ in der Basis dieser Energieeigenfunktionen:
\[
x_{mn} \;=\;\langle \Psi_m \,|\, \hat{x} \,|\, \Psi_n \rangle \;=\; 
\int_{0}^{a} \Psi_m^*(x)\; x \; \Psi_n(x)\,\mathrm{d}x
\]
vollständig auszurechnen. Im Folgenden werden alle Zwischenschritte kommentiert.

\bigskip

\subsection*{1. Definition der Matrixelemente}

Die Matrixelemente des Ortsoperators in der Energieeigenbasis sind definiert durch
\begin{equation}
x_{mn} \;=\; \int_{0}^{a} \Psi_m(x)\,x\,\Psi_n(x)\,\mathrm{d}x
\;=\; \int_{0}^{a} \left(\sqrt{\frac{2}{a}}\sin\!\biggl(\tfrac{m\pi x}{a}\biggr)\right)
\,x\,\left(\sqrt{\frac{2}{a}}\sin\!\biggl(\tfrac{n\pi x}{a}\biggr)\right)\mathrm{d}x.
\label{eq:matrixelement-definition}
\end{equation}
% Kommentar: Beide Wellenfunktionen sind reell, daher entfällt das komplexe Konjugieren.

Damit folgt
\[
x_{mn} 
= \frac{2}{a} \int_{0}^{a} x \,\sin\!\biggl(\frac{m\pi x}{a}\biggr)\,\sin\!\biggl(\frac{n\pi x}{a}\biggr)\,\mathrm{d}x.
\]

\subsection*{2. Produkt-zu-Summen-Formel für \(\sin(\alpha)\sin(\beta)\)}

% Kommentar: Um das Integral handhabbar zu machen, verwenden wir die Identität
Wir nutzen die trigonometrische Identität
\[
\sin(\alpha)\,\sin(\beta)
\;=\;\frac{1}{2}\bigl[\cos(\alpha - \beta) \;-\; \cos(\alpha + \beta)\bigr].
\]
Setzen wir \(\alpha = \tfrac{m\pi x}{a}\) und \(\beta = \tfrac{n\pi x}{a}\), so erhalten wir
\[
\sin\!\bigl(\tfrac{m\pi x}{a}\bigr)\,\sin\!\bigl(\tfrac{n\pi x}{a}\bigr)
\;=\;\frac{1}{2}\Bigl[\cos\!\bigl(\tfrac{(m-n)\pi x}{a}\bigr) \;-\; \cos\!\bigl(\tfrac{(m+n)\pi x}{a}\bigr)\Bigr].
\]
% Kommentar: Dadurch zerlegt sich das Produkt der Sinusfunktionen in Differenz von Kosinusfunktionen.

\medskip

Daher lässt sich das Matrixelement (\ref{eq:matrixelement-definition}) schreiben als
\[
x_{mn} 
= \frac{2}{a} \int_{0}^{a} x \,\frac{1}{2}\Bigl[\cos\!\bigl(\tfrac{(m-n)\pi x}{a}\bigr) - \cos\!\bigl(\tfrac{(m+n)\pi x}{a}\bigr)\Bigr]\mathrm{d}x,
\]
also
\begin{equation}
x_{mn} 
= \frac{1}{a}\Biggl[
\int_{0}^{a} x\,\cos\!\bigl(\tfrac{(m-n)\pi x}{a}\bigr)\mathrm{d}x
\;-\;
\int_{0}^{a} x\,\cos\!\bigl(\tfrac{(m+n)\pi x}{a}\bigr)\mathrm{d}x
\Biggr].
\label{eq:xmn-zwei-integrale}
\end{equation}

\subsection*{3. Spezialfall \(m=n\): Diagonalelement \(x_{nn}\)}

% Kommentar: Zunächst behandeln wir den Fall \(m = n\) separat, da \(\cos\bigl(\tfrac{(m-n)\pi x}{a}\bigr)\) dann \(\cos(0)=1\) wird.
Für \(m = n\) vereinfacht sich (\ref{eq:xmn-zwei-integrale}) zu
\[
x_{nn} 
= \frac{1}{a}\Biggl[
\int_{0}^{a} x\,\underbrace{\cos(0)}_{=1}\,\mathrm{d}x
\;-\;
\int_{0}^{a} x\,\cos\!\bigl(\tfrac{2n\pi x}{a}\bigr)\mathrm{d}x
\Biggr].
\]
\begin{itemize}
  \item[\textbullet] \textbf{Erstes Integral:}
  \[
  \int_{0}^{a} x\,\mathrm{d}x 
  = \left.\frac{x^{2}}{2}\right|_{0}^{a} 
  = \frac{a^{2}}{2}.
  \]
  % Kommentar: Das ist das einfache Integral eines Polynomterms.

  \item[\textbullet] \textbf{Zweites Integral:}
  \[
  \int_{0}^{a} x\,\cos\!\bigl(\tfrac{2n\pi x}{a}\bigr)\mathrm{d}x.
  \]
  Hierfür verwenden wir die allgemeine Stammfunktion:
  \[
  \int x\,\cos(kx)\,\mathrm{d}x 
  = x\,\frac{\sin(kx)}{k} \;+\; \frac{\cos(kx)}{k^{2}} 
  \quad(\text{plus Konstante}).
  \]
  In unserem Fall ist \(k = \tfrac{2n\pi}{a}\). Also folgt
  \[
  \int x\,\cos\!\bigl(\tfrac{2n\pi x}{a}\bigr)\mathrm{d}x
  = x\,\frac{\sin\bigl(\tfrac{2n\pi x}{a}\bigr)}{\tfrac{2n\pi}{a}}
  \;+\; \frac{\cos\bigl(\tfrac{2n\pi x}{a}\bigr)}{\bigl(\tfrac{2n\pi}{a}\bigr)^{2}}.
  \]
  Wir werten diese Stammfunktion von \(x=0\) bis \(x=a\) aus:
  \[
  \left[x\,\frac{a}{2n\pi}\,\sin\!\bigl(\tfrac{2n\pi x}{a}\bigr)
  \;+\;
  \frac{a^{2}}{(2n\pi)^{2}}\,\cos\!\bigl(\tfrac{2n\pi x}{a}\bigr)\right]_{0}^{a}.
  \]
  \begin{itemize}
    \item Bei \(x = a\):
    \[
    x\,\frac{a}{2n\pi}\,\sin\!\bigl(\tfrac{2n\pi a}{a}\bigr)
    + \frac{a^{2}}{(2n\pi)^{2}}\,\cos\!\bigl(2n\pi\bigr)
    = a\,\frac{a}{2n\pi}\,\sin(2n\pi) 
    + \frac{a^{2}}{4n^{2}\pi^{2}}\,\underbrace{\cos(2n\pi)}_{=1}.
    \]
    Da \(\sin(2n\pi)=0\), verbleibt
    \[
    \frac{a^{2}}{4n^{2}\pi^{2}}.
    \]
    \item Bei \(x = 0\):
    \[
    0\cdot\frac{a}{2n\pi}\,\sin(0) 
    + \frac{a^{2}}{4n^{2}\pi^{2}}\,\cos(0)
    = 0 + \frac{a^{2}}{4n^{2}\pi^{2}}\cdot 1 
    = \frac{a^{2}}{4n^{2}\pi^{2}}.
    \]
  \end{itemize}
  Damit ist
  \[
  \int_{0}^{a} x\,\cos\!\bigl(\tfrac{2n\pi x}{a}\bigr)\mathrm{d}x
  = \frac{a^{2}}{4n^{2}\pi^{2}} \;-\; \frac{a^{2}}{4n^{2}\pi^{2}} 
  = 0.
  \]
  % Kommentar: Das Kosinusintegral verschwindet aufgrund der periodischen Nullstellen.
\end{itemize}

Setzen wir beide Ergebnisse zusammen, ergibt sich für \(m=n\):
\[
x_{nn} 
= \frac{1}{a}\Bigl[\tfrac{a^{2}}{2} - 0\Bigr]
= \frac{a}{2}.
\]
% Zusammenfassung: Für die diagonalen Matrixelemente gilt \(x_{nn} = \frac{a}{2}\).

\bigskip

\subsection*{4. Fall \(m \neq n\): Off-Diagonalelemente \(x_{mn}\)}

Für \(m \neq n\) benutzen wir (\ref{eq:xmn-zwei-integrale}) direkt. Es stehen uns zwei Integrale der Form
\[
I_{k} \;=\; \int_{0}^{a} x\,\cos\!\bigl(\tfrac{k\pi x}{a}\bigr)\mathrm{d}x
\]
zur Verfügung, wobei \(k = m-n\) bzw.\ \(k = m+n\). Wir berechnen beide allgemein.

\smallskip

\paragraph{4.1 Allgemeines Integral \(\displaystyle \int_{0}^{a} x\,\cos\!\bigl(\tfrac{k\pi x}{a}\bigr)\mathrm{d}x\) für \(k \neq 0\).} 

Wieder verwenden wir die schon angegebene Stammfunktion:
\[
\int x\,\cos(k' x)\,\mathrm{d}x
= x\,\frac{\sin(k' x)}{k'} + \frac{\cos(k' x)}{k'^{2}}.
\]
Setze hier \(k' = \tfrac{k\pi}{a}\). Dann
\[
\int x\,\cos\!\bigl(\tfrac{k\pi x}{a}\bigr)\mathrm{d}x
= x\,\frac{\sin\bigl(\tfrac{k\pi x}{a}\bigr)}{\tfrac{k\pi}{a}}
+ \frac{\cos\bigl(\tfrac{k\pi x}{a}\bigr)}{\bigl(\tfrac{k\pi}{a}\bigr)^{2}}.
\]
Wir werten von \(x=0\) bis \(x=a\) aus:
\[
\left[x\,\frac{a}{k\pi}\,\sin\!\bigl(\tfrac{k\pi x}{a}\bigr)
+ \frac{a^{2}}{(k\pi)^{2}}\,\cos\!\bigl(\tfrac{k\pi x}{a}\bigr)\right]_{0}^{a}.
\]
\begin{itemize}
  \item Bei \(x = a\): 
  \[
  a\,\frac{a}{k\pi}\,\sin\!\bigl(\tfrac{k\pi a}{a}\bigr)
  + \frac{a^{2}}{k^{2}\pi^{2}}\,\cos\!\bigl(k\pi\bigr)
  = \frac{a^{2}}{k\pi}\,\sin(k\pi) 
  + \frac{a^{2}}{k^{2}\pi^{2}}\,(-1)^{k}.
  \]
  Da \(\sin(k\pi) = 0\) für ganzes \(k\) ist, bleibt
  \[
  \frac{a^{2}}{k^{2}\pi^{2}}\,(-1)^{k}.
  \]
  \item Bei \(x = 0\): 
  \[
  0\cdot\frac{a}{k\pi}\,\sin(0) 
  + \frac{a^{2}}{k^{2}\pi^{2}}\,\cos(0)
  = 0 + \frac{a^{2}}{k^{2}\pi^{2}}\cdot 1 
  = \frac{a^{2}}{k^{2}\pi^{2}}.
  \]
\end{itemize}
Damit ergibt sich
\[
\int_{0}^{a} x\,\cos\!\bigl(\tfrac{k\pi x}{a}\bigr)\mathrm{d}x
= \frac{a^{2}}{k^{2}\pi^{2}}\Bigl[(-1)^{k} - 1\Bigr].
\]
% Kommentar: Für \(k \neq 0\) erhalten wir diese kompakte Ausdrucksform.

\medskip

\paragraph{4.2 Einsetzen in \(x_{mn}\) für \(m \neq n\).}

Wir setzen nun \(k_1 = m - n\) und \(k_2 = m + n\) in Gleichung (\ref{eq:xmn-zwei-integrale}) ein:
\[
x_{mn}
= \frac{1}{a}\Biggl[
\underbrace{\int_{0}^{a} x\,\cos\!\bigl(\tfrac{(m-n)\pi x}{a}\bigr)\mathrm{d}x}_{I_{m-n}}
\;-\;
\underbrace{\int_{0}^{a} x\,\cos\!\bigl(\tfrac{(m+n)\pi x}{a}\bigr)\mathrm{d}x}_{I_{m+n}}
\Biggr].
\]
Daraus folgt:
\[
x_{mn}
= \frac{1}{a}\Biggl[
\frac{a^{2}}{(m-n)^{2}\pi^{2}}\bigl[\,(-1)^{\,m-n} - 1\bigr]
\;-\;
\frac{a^{2}}{(m+n)^{2}\pi^{2}}\bigl[\,(-1)^{\,m+n} - 1\bigr]
\Biggr].
\]
% Kommentar: Wir haben die Ergebnisse aus Paragraph 4.1 verwendet.

Vereinfachen wir den Faktor \(\frac{a^{2}}{a} = a\) und notieren 
\(\,s_{mn} = (-1)^{\,m+n}.\)
\[
x_{mn}
= \frac{a}{\pi^{2}}\Biggl[
\frac{(-1)^{\,m-n} - 1}{(m-n)^{2}}
\;-\;
\frac{(-1)^{\,m+n} - 1}{(m+n)^{2}}
\Biggr].
\]
\emph{Wichtig:} Da \(\;(-1)^{\,m-n} = (-1)^{\,m}\,(-1)^{-n} 
= (-1)^{\,m}\,(-1)^{\,n} = (-1)^{\,m+n} = s_{mn},\) folgt
\[
(-1)^{\,m-n} \;=\; (-1)^{\,m+n} \;=\; s_{mn}.
\]
Somit wird der obige Ausdruck:
\begin{align*}
x_{mn}
&= \frac{a}{\pi^{2}}\Biggl[
\frac{s_{mn} - 1}{(m-n)^{2}}
\;-\;
\frac{s_{mn} - 1}{(m+n)^{2}}
\Biggr]\\
&= \frac{a\,(s_{mn} - 1)}{\pi^{2}}
\Biggl[
\frac{1}{(m-n)^{2}} - \frac{1}{(m+n)^{2}}
\Biggr].
\end{align*}
% Kommentar: Jetzt bündeln wir den gemeinsamen Faktor \((s_{mn}-1)\).

\medskip

\paragraph{4.3 Vereinfachung der Differenz \(\frac{1}{(m-n)^{2}} - \frac{1}{(m+n)^{2}}\).}

Betrachte
\[
\frac{1}{(m-n)^{2}} \;-\; \frac{1}{(m+n)^{2}}
= \frac{(m+n)^{2} - (m-n)^{2}}{(m-n)^{2}\,(m+n)^{2}}.
\]
Berechne den Zähler:
\[
(m+n)^{2} - (m-n)^{2} 
= \bigl(m^{2} + 2mn + n^{2}\bigr) \;-\; \bigl(m^{2} - 2mn + n^{2}\bigr)
= 4\,m\,n.
\]
Daher
\[
\frac{1}{(m-n)^{2}} - \frac{1}{(m+n)^{2}}
= \frac{4\,m\,n}{(m-n)^{2}\,(m+n)^{2}}
= \frac{4\,m\,n}{(m^{2} - n^{2})^{2}}.
\]
% Kommentar: Die Differenz zweier Kehrwerte liefert einen Faktor \(4mn\) im Zähler.

Somit folgt
\[
x_{mn}
= \frac{a\,(s_{mn} - 1)}{\pi^{2}}
\;\cdot\;
\frac{4\,m\,n}{(m^{2} - n^{2})^{2}}
\;=\;
\frac{4\,a\,m\,n\,(s_{mn} - 1)}{\pi^{2}\,(m^{2} - n^{2})^{2}}.
\]
Wegen \(s_{mn} = (-1)^{\,m+n}\) können wir das Vorzeichen weiter spezifizieren:
\begin{itemize}
  \item Wenn \(m + n\) \emph{gerade} ist, dann ist \(s_{mn} = (+1)\). Dann \(s_{mn} - 1 = 0\), also
  \[
  x_{mn} = 0 \quad\text{für \(m+n\) gerade.}
  \]
  \item Wenn \(m + n\) \emph{ungerade} ist, dann ist \(s_{mn} = (-1)\). Dann \(s_{mn} - 1 = -2\), und somit
  \[
  x_{mn}
  = \frac{4\,a\,m\,n\,(-2)}{\pi^{2}\,(m^{2} - n^{2})^{2}}
  = -\,\frac{8\,a\,m\,n}{\pi^{2}\,(m^{2} - n^{2})^{2}}
  \quad\text{für \(m+n\) ungerade.}
  \]
\end{itemize}
% Kommentar: Damit sind die Off-Diagonalelemente nicht verschwunden, wenn \(m+n\) ungerade ist.

Wir fassen die Ergebnisse zusammen:
\[
x_{mn} \;=\;
\begin{cases}
\dfrac{a}{2}, & m = n,\\[1em]
0, & m \neq n \text{ und } m+n \text{ gerade},\\[1em]
-\,\dfrac{8\,a\,m\,n}{\pi^{2}\,(m^{2} - n^{2})^{2}}, 
& m \neq n \text{ und } m+n \text{ ungerade}.
\end{cases}
\]

\bigskip

\subsection*{5. Endgültiges Resultat und Interpretation}

Die Matrixdarstellung des Ortsoperators $\hat{x}$ in der Basis $\{\Psi_n\}$ lautet demnach:
\[
\boxed{
x_{mn} \;=\; 
\langle \Psi_m | \hat{x} | \Psi_n \rangle
=
\begin{cases}
\dfrac{a}{2}, 
& m = n,\\[1em]
0, 
& m \neq n \text{ und } m+n \text{ gerade},\\[1em]
-\,\dfrac{8\,a\,m\,n}{\pi^{2}\,(m^{2} - n^{2})^{2}}, 
& m \neq n \text{ und } m+n \text{ ungerade}.
\end{cases}
}
\]
\begin{itemize}
  \item \textbf{Diagonalelemente:} $x_{nn} = \dfrac{a}{2}$ entspricht dem Mittelwert $\langle x \rangle$ im Zustand $\Psi_n$.
  \item \textbf{Off-Diagonalelemente:} Nur Zustände unterschiedlicher Parität (ungerade Summe \(m+n\)) koppeln sich. Die Größe der Kopplung fällt mit \(\propto 1/(m^{2}-n^{2})^{2}\) sehr schnell ab.
\end{itemize}


\end{document}
