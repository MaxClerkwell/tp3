\documentclass[a4paper,12pt]{article}
\usepackage{amsmath,amsfonts,amssymb}
\usepackage{geometry}
\geometry{margin=1in}
\usepackage{physics}
\usepackage{siunitx}
\AtBeginDocument{\RenewCommandCopy\qty\SI} % Fix siunitx/physics conflict
\usepackage{enumitem}
\usepackage{mathrsfs}
\usepackage[T1]{fontenc}
\usepackage{lmodern}
\usepackage{nicefrac}
\usepackage{breqn} % For automatic equation breaking
\usepackage{microtype} % Improves typography and reduces overfull boxes

\begin{document}

\title{Quantenmechanik (Sommersemester 2025) \\ Hausaufgabe 8 Lösungen}
\author{Stephan Bökelmann, Meihui Huang}
\date{June 20, 2025}
\maketitle

\section*{H8.1 Zustandsraum endlicher Dimension}

\subsection*{a) Hermitesche Operatoren und gemeinsame Diagonalisierung}

To determine if the operators $\hat{H}$, $\hat{A}$, and $\hat{B}$ are Hermitian, we check if their matrix representations satisfy $\boldsymbol{M}^\dagger = \boldsymbol{M}$. For a matrix $\boldsymbol{M}$, the adjoint is the complex conjugate transpose, $\boldsymbol{M}^\dagger = (\boldsymbol{M}^*)^T$.

For $\boldsymbol{H} = \hbar \omega \begin{pmatrix} 1 & 0 & 0 \\ 0 & 1 & 0 \\ 0 & 0 & -1 \end{pmatrix}$, since it is real and diagonal, $\boldsymbol{H}^\dagger = \boldsymbol{H}$, so $\hat{H}$ is Hermitian.

For $\boldsymbol{A} = a \begin{pmatrix} 0 & 1 & 0 \\ 1 & 0 & 0 \\ 0 & 0 & -1 \end{pmatrix}$, compute the adjoint:
\[
\boldsymbol{A}^\dagger = a \begin{pmatrix} 0 & 1 & 0 \\ 1 & 0 & 0 \\ 0 & 0 & -1 \end{pmatrix}^\dagger = a \begin{pmatrix} 0 & 1 & 0 \\ 1 & 0 & 0 \\ 0 & 0 & -1 \end{pmatrix} = \boldsymbol{A},
\]
since the matrix is real and symmetric. Thus, $\hat{A}$ is Hermitian.

For $\boldsymbol{B} = b \begin{pmatrix} 1 & 0 & 0 \\ 0 & 0 & 1 \\ 0 & 1 & 0 \end{pmatrix}$, compute:
\[
\boldsymbol{B}^\dagger = b \begin{pmatrix} 1 & 0 & 0 \\ 0 & 0 & 1 \\ 0 & 1 & 0 \end{pmatrix}^\dagger = b \begin{pmatrix} 1 & 0 & 0 \\ 0 & 0 & 1 \\ 0 & 1 & 0 \end{pmatrix} = \boldsymbol{B},
\]
since it is also real and symmetric. Thus, $\hat{B}$ is Hermitian.

Since $\hat{H}$ and $\hat{A}$ commute, they can be simultaneously diagonalized. The matrix $\boldsymbol{H}$ is already diagonal with eigenvalues $\hbar \omega, \hbar \omega, -\hbar \omega$. Compute the eigenvalues of $\boldsymbol{A}$:
\begin{dmath}
\det(\boldsymbol{A} - \lambda \boldsymbol{I}) = \det \begin{pmatrix} -\lambda & a & 0 \\ a & -\lambda & 0 \\ 0 & 0 & -a - \lambda \end{pmatrix}
= (-\lambda)(-\lambda)(-a - \lambda) - a(a)(-a - \lambda)
= -(\lambda^2 - a^2)(-a - \lambda).
\end{dmath}
Roots are $\lambda = a, -a, -a$. The eigenspace for $\lambda = -a$ is two-dimensional (spanned by $\ket{\Phi_2}$ and $\ket{\Phi_3}$), and for $\lambda = a$, it is one-dimensional (spanned by $\ket{\Phi_1} + \ket{\Phi_2}$). Since $\boldsymbol{H}$ is diagonal in the basis $\{\ket{\Phi_1}, \ket{\Phi_2}, \ket{\Phi_3}\}$, and $\boldsymbol{A}$ shares eigenvectors with $\boldsymbol{H}$ in this basis (after choosing appropriate eigenvectors for the degenerate eigenvalue), the basis $\{\ket{\Phi_1}, \ket{\Phi_2}, \ket{\Phi_3}\}$ already diagonalizes both. Thus, the transformation matrix is the identity:
\[
\boldsymbol{U} = \begin{pmatrix} 1 & 0 & 0 \\ 0 & 1 & 0 \\ 0 & 0 & 1 \end{pmatrix}.
\]

\subsection*{b) Energiemessung bei $t=0$}

The state at $t=0$ is $\ket{\Psi} = \frac{1}{2} \ket{\Phi_1} + \frac{1}{2} \ket{\Phi_2} + \frac{1}{\sqrt{2}} \ket{\Phi_3}$. The eigenvalues of $\hat{H}$ are $E_1 = E_2 = \hbar \omega$ (for $\ket{\Phi_1}, \ket{\Phi_2}$) and $E_3 = -\hbar \omega$ (for $\ket{\Phi_3}$). Probabilities are:
\[
P(E = \hbar \omega) = \left| \braket{\Phi_1}{\Psi} \right|^2 + \left| \braket{\Phi_2}{\Psi} \right|^2 = \left( \frac{1}{2} \right)^2 + \left( \frac{1}{2} \right)^2 = \frac{1}{4} + \frac{1}{4} = \frac{1}{2},
\]
\[
P(E = -\hbar \omega) = \left| \braket{\Phi_3}{\Psi} \right|^2 = \left( \frac{1}{\sqrt{2}} \right)^2 = \frac{1}{2}.
\]
Expectation value:
\[
\langle \hat{H} \rangle = \sum_i P(E_i) E_i = \frac{1}{2} (\hbar \omega) + \frac{1}{2} (-\hbar \omega) = 0.
\]
For the uncertainty, compute $\langle \hat{H}^2 \rangle$:
\[
\boldsymbol{H}^2 = (\hbar \omega)^2 \begin{pmatrix} 1 & 0 & 0 \\ 0 & 1 & 0 \\ 0 & 0 & 1 \end{pmatrix}, \quad \langle \hat{H}^2 \rangle = \bra{\Psi} \boldsymbol{H}^2 \ket{\Psi} = (\hbar \omega)^2 \left( \frac{1}{4} + \frac{1}{4} + \frac{1}{2} \right) = (\hbar \omega)^2.
\]
\[
(\Delta E)^2 = \langle \hat{H}^2 \rangle - \langle \hat{H} \rangle^2 = (\hbar \omega)^2 - 0 = (\hbar \omega)^2, \quad \Delta E = \hbar \omega.
\]

\subsection*{c) Messung der Observable $A$ bei $t=0$}

Eigenvalues of $\hat{A}$ are $a$ (eigenvector $\frac{1}{\sqrt{2}} (\ket{\Phi_1} + \ket{\Phi_2})$) and $-a$ (eigenvectors $\ket{\Phi_2}, \ket{\Phi_3}$). Probabilities:
\[
P(A = a) = \left| \bra{\Psi_a} \Psi \right|^2, \quad \ket{\Psi_a} = \frac{1}{\sqrt{2}} (\ket{\Phi_1} + \ket{\Phi_2}), \quad \bra{\Psi_a} \Psi = \frac{1}{\sqrt{2}} \left( \frac{1}{2} + \frac{1}{2} \right) = \frac{1}{\sqrt{2}},
\]
\[
P(A = a) = \left| \frac{1}{\sqrt{2}} \right|^2 = \frac{1}{2}.
\]
\[
P(A = -a) = \left| \braket{\Phi_2}{\Psi} \right|^2 + \left| \braket{\Phi_3}{\Psi} \right|^2 = \frac{1}{4} + \frac{1}{2} = \frac{3}{4}.
\]
Post-measurement states: For $A = a$, the state collapses to $\ket{\Psi_a}$. For $A = -a$, the state collapses to a linear combination, e.g., $\sqrt{\frac{2}{3}} \ket{\Phi_3}$.

\subsection*{d) Zeitentwicklung des Zustandsvektors}

The time evolution is $\ket{\Psi(t)} = e^{-i \hat{H} t / \hbar} \ket{\Psi}$. Since $\boldsymbol{H}$ is diagonal:
\[
e^{-i \boldsymbol{H} t / \hbar} = \begin{pmatrix} e^{-i \omega t} & 0 & 0 \\ 0 & e^{-i \omega t} & 0 \\ 0 & 0 & e^{i \omega t} \end{pmatrix}.
\]
\[
\ket{\Psi(t)} = \frac{1}{2} e^{-i \omega t} \ket{\Phi_1} + \frac{1}{2} e^{-i \omega t} \ket{\Phi_2} + \frac{1}{\sqrt{2}} e^{i \omega t} \ket{\Phi_3}.
\]

\subsection*{e) Erwartungswerte von $A$ und $B$}

For $\hat{A}$:
\begin{dmath}
\langle \hat{A} \rangle(t) = \bra{\Psi(t)} \boldsymbol{A} \ket{\Psi(t)}
= a \left( \frac{1}{2} e^{i \omega t} \cdot \frac{1}{2} e^{-i \omega t} + \frac{1}{2} e^{-i \omega t} \cdot \frac{1}{2} e^{i \omega t} - \frac{1}{\sqrt{2}} e^{-i \omega t} \cdot \frac{1}{\sqrt{2}} e^{i \omega t} \right)
= a \left( \frac{1}{4} + \frac{1}{4} - \frac{1}{2} \right) = 0.
\end{dmath}
For $\hat{B}$:
\begin{dmath}
\langle \hat{B} \rangle(t) = b \left( \frac{1}{4} + \frac{1}{2} e^{-2i \omega t} + \frac{1}{2} e^{2i \omega t} \right)
= b \left( \frac{1}{4} + \cos(2 \omega t) \right).
\end{dmath}
Observation: $\langle \hat{A} \rangle$ is time-independent, consistent with $[\hat{H}, \hat{A}] = 0$. $\langle \hat{B} \rangle$ oscillates, as $[\hat{H}, \hat{B}] \neq 0$.

\subsection*{f) Messungen bei $t \neq 0$}

For $\hat{A}$, probabilities are time-independent due to $[\hat{H}, \hat{A}] = 0$. For $\hat{B}$, compute probabilities using $\ket{\Psi(t)}$ and the eigenvectors of $\boldsymbol{B}$, which yield time-dependent results due to non-commutation.

\end{document}
