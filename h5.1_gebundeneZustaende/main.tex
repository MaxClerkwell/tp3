\documentclass[a4paper,12pt]{article}
\usepackage[utf8]{inputenc}
\usepackage{amsmath,amssymb}
\usepackage{graphicx}
\usepackage{geometry}
\geometry{margin=2.5cm}

\begin{document}

\title{Lösung zu Aufgabe H5.1: Existenz gebundener Zustände}
\author{Stephan Bökelmann, Meihui Huang}
\date{\today}
\maketitle

\section*{Aufgabenstellung}
Ein Teilchen der Masse $m$ bewege sich in einem eindimensionalen, kastenförmigen Potential
\[
V(x)=
\begin{cases}
\infty,& x\le0,\\
-\,V_0,& 0<x<a,\\
0,& x\ge a,
\end{cases}
\]
wobei $V_0>0$ ist.

\begin{enumerate}
  \item[(a)] Stellen Sie die allgemeinen Lösungen der stationären Schrödingergleichung in den
    jeweiligen Bereichen für \emph{positive} und \emph{negative} Energieeigenwerte auf.
  \item[(b)] Formulieren Sie die Randbedingungen bei $x=0$, $x=a$ und $x\to\infty$ und wenden Sie
    diese an. Leiten Sie damit die Bedingung für die Konstante $V_0$ her, unter der gebundene
    Zustände (Energie $E<0$) existieren.
\end{enumerate}

\section{(a) Allgemeine Lösungen der Schrödingergleichung}

Die zeitunabhängige Schrödingergleichung lautet
\[
-\frac{\hbar^2}{2m}\,\psi''(x) + V(x)\,\psi(x) = E\,\psi(x).
\]
Wir betrachten getrennt die drei Bereiche:

\subsection*{Bereich I: \boldmath $x\le0$}
\[
V(x)=\infty \quad\Longrightarrow\quad \psi_I(x)=0
\quad\text{(Zwangsbedingung durch unendliche Mauer).}
\]

\subsection*{Bereich II: \boldmath $0<x<a$, Potential $V=-V_0$}

\begin{itemize}
  \item \textbf{Für positive Energien $E>0$:}
    \[
      E - V(x) = E + V_0 > V_0,
    \]
    wir setzen
    \[
      k = \sqrt{\frac{2m\,(E+V_0)}{\hbar^2}} > 0.
    \]
    Die Gleichung lautet
    \[
    \psi_{II}''(x) + k^2\,\psi_{II}(x) = 0,
    \]
    mit allgemeiner Lösung
    \[
      \psi_{II}(x) = A\,\sin(kx) + B\,\cos(kx).
      \quad % A, B: Integrationskonstanten
    \]
  \item \textbf{Für negative Energien $E<0$:} Für gebundene Zustände nehmen wir $-V_0<E<0$,
    also $E+V_0>0$. Setze erneut
    \[
      k = \sqrt{\frac{2m\,(E+V_0)}{\hbar^2}}>0,
    \]
    und erhält dieselbe Form
    \[
      \psi_{II}(x) = A\,\sin(kx) + B\,\cos(kx).
    \]
\end{itemize}

\subsection*{Bereich III: \boldmath $x\ge a$, Potential $V=0$}

\begin{itemize}
  \item \textbf{Für $E>0$:} Setze
    \[
      k' = \sqrt{\frac{2m\,E}{\hbar^2}}>0,
    \]
    die Gleichung
    \[
      \psi_{III}''(x) + k'^2 \,\psi_{III}(x) = 0
    \]
    löst sich allgemein zu
    \[
      \psi_{III}(x) = C\,e^{i k' x} + D\,e^{-i k' x}.
      \quad % Wellen rein/ausgehend
    \]
  \item \textbf{Für gebundene Zustände $E<0$:} Schreibe $E=-|E|<0$ und setze
    \[
      \kappa = \sqrt{\frac{2m\,|E|}{\hbar^2}}>0.
    \]
    Dann
    \[
      \psi_{III}''(x) - \kappa^2\,\psi_{III}(x) = 0
      \quad\Longrightarrow\quad
      \psi_{III}(x) = C\,e^{\kappa x} + D\,e^{-\kappa x}.
    \]
\end{itemize}

\section{(b) Randbedingungen und Existenzbedingung für gebundene Zustände}

\subsection*{Randbedingung bei \boldmath $x=0$}

\[
\psi_I(0)=0 \quad\overset{\psi_I=0}{\Longrightarrow}\quad
\psi_{II}(0)=0.
\]
Aus $\psi_{II}(0)=A\sin(0)+B\cos(0)=B=0$ folgt
\[
B=0,\quad
\psi_{II}(x)=A\,\sin(kx).
\]

\subsection*{Randbedingungen bei \boldmath $x=a$ und für \boldmath $x\to\infty$}

\begin{itemize}
  \item \emph{Stetigkeit der Wellenfunktion:}
    \[
      \psi_{II}(a) = \psi_{III}(a).
      \quad\text{(1)}
    \]
  \item \emph{Stetigkeit der Ableitung:}
    \[
      \psi_{II}'(a) = \psi_{III}'(a).
      \quad\text{(2)}
    \]
  \item \emph{Normintegrabilität für gebundene Zustände:}
    \[
      \psi_{III}(x)\;\xrightarrow[x\to\infty]{}\;0
      \quad\Longrightarrow\quad C=0
      \quad\text{(den wachsenden Anteil verwerfen).}
    \]
    Damit bleibt
    \[
      \psi_{III}(x)=D\,e^{-\kappa x}.
    \]
\end{itemize}

\subsection*{Herleitung der Transzendenzbedingung}

Setze nun
\[
\psi_{II}(x)=A\sin(kx),\quad
\psi_{III}(x)=D\,e^{-\kappa x}.
\]
\begin{align*}
  \text{(1):}\quad & A\sin(k a) = D\,e^{-\kappa a}, \\
  \text{(2):}\quad & A\,k\cos(k a) = -D\,\kappa\,e^{-\kappa a}.
\end{align*}
Dividiere (2) durch (1):
\[
\frac{k\cos(k a)}{\sin(k a)}
= -\,\kappa
\quad\Longrightarrow\quad
k\,\cot(k a) = -\,\kappa.
\]
\[
\boxed{\,k\,\cot(k a) = -\,\kappa.}
\]
Erinnere an die Definitionen:
\[
k = \sqrt{\frac{2m\,(E+V_0)}{\hbar^2}}, 
\quad
\kappa = \sqrt{\frac{2m\,|E|}{\hbar^2}}
= \sqrt{\frac{2m\,(-E)}{\hbar^2}}.
\]
Mit $E=-|E|$ wird die Bedingung
\[
\sqrt{E+V_0}\,\cot\!\bigl(a\sqrt{2m\,(E+V_0)}/\hbar\bigr)
= -\,\sqrt{-E},
\]
was man auch als
\[
\cot(\alpha)\;=\;-\,\frac{\sqrt{-E}}{\sqrt{E+V_0}}
\quad\text{mit}\quad
\alpha = \frac{a}{\hbar}\sqrt{2m\,(E+V_0)}
\]
schreiben kann.

\subsection*{Existenzbedingung für mindestens einen gebundenen Zustand}

Für das Intervall $0<k a<\pi$ (erste Nullstelle von $\sin$) muss $\cot(k a)$ positiv sein,
damit die Gleichung gelöst werden kann. Also
\[
0 < k a < \frac{\pi}{2}
\quad\Longrightarrow\quad
k a < \frac{\pi}{2}.
\]
Setze $E\to 0^-$ (Grenzfall), dann $k^2 \to \tfrac{2mV_0}{\hbar^2}$:
\[
k = \sqrt{\frac{2mV_0}{\hbar^2}},
\quad
k a < \frac{\pi}{2}
\;\Longrightarrow\;
\sqrt{\frac{2mV_0}{\hbar^2}}\,a < \frac{\pi}{2}.
\]
Daraus folgt die minimale Tiefe $V_0$ für das Auftreten des ersten gebundenen Zustands:
\[
V_0 > \frac{\pi^2\hbar^2}{8 m a^2}.
\]
\[
\boxed{\,V_0^{\rm (kritisch)} = \frac{\pi^2\hbar^2}{8 m a^2}.}
\]
Für $V_0$ kleiner als dieser kritischen Tiefe existiert kein gebundener Zustand; für $V_0$ größer
als dieser kritischen Tiefe existiert mindestens ein gebundener Zustand.

\section*{Zusammenfassung}

\begin{itemize}
  \item Im link unendlichen Rand ($x\le0$) ist $\psi=0$.
  \item Im Potentialkasten ($0<x<a$) sind die Lösungen Sinus- und Kosinusfunktionen mit Impuls
    $k=\sqrt{2m\,(E+V_0)}/\hbar$.
  \item Rechts vom Kasten ($x\ge a$) fällt die Wellenfunktion exponentiell ab, $\exp(-\kappa x)$,
    mit Zerfallslänge $\kappa=\sqrt{2m(-E)}/\hbar$.
  \item Die Matching-Bedingungen an $x=a$ liefern die Transzendenzgleichung
    $k\cot(k a)=-\kappa$.
  \item Die Bedingung für das erste gebundene Niveau ergibt:
    $V_0 > \pi^2\hbar^2/(8 m a^2)$.
\end{itemize}

\end{document}
