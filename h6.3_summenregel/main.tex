
\documentclass[12pt]{scrartcl}
% Paket für erweiterte mathematische Umgebungen
\usepackage{amsmath,amssymb}
\usepackage{physics} % Für bequeme Notation von Kommutatoren etc.
\usepackage[a4paper,margin=2.5cm]{geometry}

\begin{document}

\section*{Lösung zu H6.3: Summenregel}

\textbf{Aufgabenstellung.}
Seien $\{\ket{\Psi_0},\ket{\Psi_1},\dots,\ket{\Psi_k}\}$ normierte Eigenzustände eines Hamiltonoperators
$\hat H$ eines Teilchens der Masse $m$ im eindimensionalen Raum, mit nicht entarteten Energieeigenwerten
$\{E_0,E_1,\dots,E_k\}$. Das Potential $V=V(\hat x)$ hängt nur vom Ortsoperator ab. Zeigen Sie, dass
\[
  \sum_{n=0}^k (E_n - E_m)\,\bigl|\langle \Psi_n | \hat x | \Psi_m \rangle\bigr|^2 = \mathrm{const.}
\]
unabhängig von der konkreten Wahl des Potentials, und bestimmen Sie die Konstante.\\

\textbf{Hinweis.} Betrachten Sie dazu den doppelten Kommutator
\(
  [[\hat H,\hat x],\hat x]
\).

\bigskip

\textbf{1. Schritt: Berechnung des ersten Kommutators}\\
Wir zerlegen den Hamiltonoperator in kinetischen und Potentialanteil:
\[
  \hat H = \frac{\hat p^2}{2m} + V(\hat x).
\]
Da $V(\hat x)$ nur vom Ort abhängt, kommutiert es mit $\hat x$:
\[
  [V(\hat x),\hat x] = 0.
\]
Damit reduziert sich
\[
  [\hat H,\hat x] = \Bigl[\frac{\hat p^2}{2m},\hat x\Bigr].
\]
Nutzen wir die bekannten Beziehungen
\[
  [\hat p,\hat x] = -i\hbar,
  \qquad
  [\hat p^2,\hat x] = \hat p\,[\hat p,\hat x] + [\hat p,\hat x]\,\hat p = -2i\hbar\,\hat p.
\]
Daraus folgt
\[
  [\hat H,\hat x] = \frac{1}{2m}(-2i\hbar\,\hat p) = -\frac{i\hbar}{m}\,\hat p.
\]

\textbf{2. Schritt: Doppelter Kommutator}\\
Nun berechnen wir den Kommutator von $[\hat H,\hat x]$ mit $\hat x$:
\[
  [[\hat H,\hat x],\hat x]
  = \Bigl[-\frac{i\hbar}{m}\,\hat p,\,\hat x\Bigr]
  = -\frac{i\hbar}{m}\,[\hat p,\hat x]
  = -\frac{i\hbar}{m}(-i\hbar)
  = -\frac{\hbar^2}{m}
  =: C.
\]
Wichtig: $C$ ist eine vom Potential unabhängige Konstante.

\textbf{3. Schritt: Zusammenhang zur Summenregel}\\
Wir betrachten nun die Erwartung des doppelten Kommutators im Zustand $\ket{\Psi_m}$:
\[
  \bigl\langle \Psi_m \bigm| [[\hat H,\hat x],\hat x] \bigm| \Psi_m \bigr\rangle
  = C
  = -\frac{\hbar^2}{m}.
\]
Auf der anderen Seite lässt sich durch Einfügen der Vollständigkeitsrelation
$\sum_n \ket{\Psi_n}\bra{\Psi_n}=\mathbb{1}$ und Ausnutzung
$\hat H\ket{\Psi_n}=E_n\ket{\Psi_n}$ zeigen (siehe Beispielrechnung am Ende), dass
\[
  \bigl\langle [[\hat H,\hat x],\hat x]\bigr\rangle
  = 2\sum_{n=0}^k (E_m - E_n)\,\bigl|x_{nm}\bigr|^2,
  \quad x_{nm}=\langle\Psi_n|\hat x|\Psi_m\rangle.
\]
Damit folgt
\[
  \sum_{n=0}^k (E_n - E_m)\,\bigl|x_{nm}\bigr|^2 = \frac{\hbar^2}{2m}
  \quad\text{(Konstante).}
\]

\textbf{4. Ausführliche Beispielrechnung}\\
Wir skizzieren kurz den Nachweis des letzten Schritts:
\begin{align*}
  [[\hat H,\hat x],\hat x]
  &= \hat H\hat x^2 - 2\hat x\hat H\hat x + \hat x^2\hat H, \\
  \bigl\langle \Psi_m \bigm|[[\hat H,\hat x],\hat x]\bigm| \Psi_m \bigr\rangle
  &= E_m\langle \Psi_m|\hat x^2|\Psi_m\rangle + E_m\langle \Psi_m|\hat x^2|\Psi_m\rangle
     -2\langle \Psi_m|\hat x\hat H\hat x|\Psi_m\rangle \\
  &= 2E_m\sum_n |x_{nm}|^2 -2\sum_n E_n|x_{nm}|^2
   = 2\sum_n (E_m-E_n)|x_{nm}|^2.
\end{align*}
Umstellen nach $\sum_n(E_n-E_m)|x_{nm}|^2$ liefert die obige Konstante.

\textbf{Hinweis zur Interpretation.} Diese Regel ist unabhängig von der Form des Potentials und ergibt eine nützliche
Identität, z.B. für Abschätzungen oder Vergleichsgrößen in Störungsrechnungen.

\end{document}
