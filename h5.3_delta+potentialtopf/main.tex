\documentclass[a4paper,12pt]{article}
\usepackage[utf8]{inputenc}
\usepackage{amsmath,amssymb}
\usepackage{geometry}
\geometry{margin=2.5cm}

\begin{document}

\title{Lösung zu Aufgabe H5.3: Delta-Potential im Potentialtopf}
\author{Stephan Bökelmann, Meihui Huang}
\date{\today}
\maketitle

\section*{Aufgabenstellung}
Ein Teilchen der Masse $m$ bewegt sich in einem eindimensionalen Topf
\[
V(x)=
\begin{cases}
\alpha\,\delta(x), & |x|<a,\\
\infty,            & |x|\ge a,
\end{cases}
\qquad \alpha>0.
\]
\begin{enumerate}
  \item[(a)] Bestimmen Sie das Spektrum des Systems. Leiten Sie die Gleichungen her, 
    deren Lösungen die erlaubten Energieeigenwerte $E=\tfrac{\hbar^2k^2}{2m}$ beschreiben.
  \item[(b)] Zeigen Sie für $m\alpha a/\hbar^2\gg1$, dass im unteren Spektralbereich
    ($ka\ll m\alpha a/\hbar^2$) die Energieeigenwerte in fast entarteten Paaren 
    auftreten, die für $a\to\infty$ beliebig nahe beieinander liegen. 
    Setzen Sie dazu $ka = n\pi - \varepsilon$ und bestimmen Sie $\varepsilon$.
\end{enumerate}

\section{(a) Spektralgleichungen}

\subsection*{1. Stationäre Schrödingergleichung}
Für $|x|<a$, ohne den $\delta$-Impuls, gilt
\[
-\frac{\hbar^2}{2m}\,\psi''(x) = E\,\psi(x),
\]
mit allgemeiner Lösung
\[
\psi(x) = 
\begin{cases}
A\,\sin(kx) + B\,\cos(kx), & -a<x<0,\\
C\,\sin(kx) + D\,\cos(kx), & 0<x<a,
\end{cases}
\qquad
k = \sqrt{\frac{2mE}{\hbar^2}}.
\]
Außen ($|x|\ge a$) ist $V=\infty$, daher
\[
\psi(\pm a) = 0.
\]

\subsection*{2. Symmetrie und Randbedingungen}
Da das Potential symmetrisch ist, zerlegen wir in gerade und ungerade Eigenfunktionen:

\paragraph*{(i) Ungerade Zustände ($\psi(-x)=-\psi(x)$):}
\[
\psi(0)=0  \quad\Longrightarrow\quad 
\psi(x) = 
\begin{cases}
A\,\sin\bigl[k(x + a)\bigr], & -a<x<0,\\
A\,\sin\bigl[k(a - x)\bigr], & 0<x<a,
\end{cases}
\]
so dass automatisch $\psi(\pm a)=0$ erfüllt ist. Keine Sprungbedingung an $x=0$, da 
$\delta(x)\psi(0)=0$. Damit erhält man die Quantisierungsbedingung
\[
\sin(k a) = 0
\;\;\Longrightarrow\;\;
k_n^{(\text{odd})} = \frac{n\pi}{a},\quad n=1,2,3,\dots
\]
und
\[
E_n^{(\text{odd})} = \frac{\hbar^2}{2m}\,\Bigl(\tfrac{n\pi}{a}\Bigr)^2.
\]

\paragraph*{(ii) Gerade Zustände ($\psi(-x)=\psi(x)$):}
\[
\psi(x) = 
\begin{cases}
B\,\cos\bigl[k(x + a)\bigr], & -a<x<0,\\
B\,\cos\bigl[k(a - x)\bigr], & 0<x<a,
\end{cases}
\]
wiederum $\psi(\pm a)=0$. Bei $x=0$ gilt $\psi(0)=B\cos(ka)\neq0$, daher wirkt die
$\delta$-Störung. Die Sprungbedingung für die Ableitung
\[
\psi'(0^+) - \psi'(0^-) \;=\;\frac{2m\alpha}{\hbar^2}\,\psi(0)
\]
führt auf
\[
-2\,k\,B\,\sin(k a)
\;=\;\frac{2m\alpha}{\hbar^2}\,B\,\cos(k a)
\quad\Longrightarrow\quad
k\,\tan(k a) \;=\;\frac{m\alpha}{\hbar^2}.
\]
Dies ist die Transzendenzbedingung für gerade Zustände:
\[
k_n^{(\text{even})}\;\text{lösbar aus}\;k\,\tan(k a)=\beta,
\quad\beta:=\frac{m\alpha}{\hbar^2}.
\]
Die zugehörigen Energien sind
\[
E_n^{(\text{even})} = \frac{\hbar^2}{2m}\,\bigl(k_n^{(\text{even})}\bigr)^2.
\]

\medskip
\noindent
\boxed{
\begin{aligned}
&\text{odd:}   &k_n &= \tfrac{n\pi}{a},\quad n=1,2,\dots, 
&&E_n=\tfrac{\hbar^2}{2m}(n\pi/a)^2,\\
&\text{even:} &k\;\tan(k a)&=\tfrac{m\alpha}{\hbar^2}, 
&&E=\tfrac{\hbar^2 k^2}{2m}\,.
\end{aligned}
}

\section{(b) Näherung für starke Störung $\beta a\gg1$}

Im unteren Spektralbereich gilt $k a \ll \beta a$, daher auch $k a \ll \beta a \implies k a \ll \beta a$:

\subsection*{1. Ungerade Zustände}
Für die ungeraden Linien $k_n^{(\text{odd})}=n\pi/a$ unverändert. 

\subsection*{2. Gerade Zustände: $k a = n\pi - \varepsilon$}
Wir setzen für das $n$-te gerade Niveau
\[
k a = n\pi - \varepsilon,\qquad 0<\varepsilon\ll1.
\]
Dann
\[
\tan(k a) = \tan(n\pi - \varepsilon) = -\,\tan(\varepsilon)\approx -\,\varepsilon.
\]
Eingesetzt in
\[
k\,\tan(k a) = \beta
\;\;\Longrightarrow\;\;
k\,(-\varepsilon) \;\approx\;\beta
\;\;\Longrightarrow\;\;
\varepsilon \;\approx\; -\,\frac{\beta}{k}
\]
Da $k\approx n\pi/a$ und $\beta\gg k$, ist $\varepsilon\gg1$ ungültig im ersten Blick. 
Stattdessen nutzt man für große $\beta a$ die Polstellen von $\tan$: 
Man schreibt besser
\[
k a = \Bigl(n+\tfrac12\Bigr)\pi + \varepsilon,
\]
weil dort $\tan$ eine Null kreuzt. Dann
\[
\tan\bigl((n+\tfrac12)\pi + \varepsilon\bigr) = -\,\cot(\varepsilon)\approx -\tfrac1\varepsilon,
\]
und die Gleichung $k\,\tan(k a)=\beta$ wird
\[
k\,\Bigl(-\frac1\varepsilon\Bigr)=\beta
\quad\Longrightarrow\quad
\varepsilon \approx \frac{k}{\beta}
\approx \frac{(n+\tfrac12)\pi/a}{m\alpha/\hbar^2}
= \frac{(n+\tfrac12)\,\pi\,\hbar^2}{m\alpha\,a}.
\]
Damit liegen die geraden Niveaus nahe bei den halbzahlig-fachen Knoten des leeren Kastens, 
verschoben um $\varepsilon$. 

\subsection*{3. Entartung der Paare}
Für $\beta a\gg1$ gilt
\[
\varepsilon \ll1,
\]
also
\[
k_n^{(\text{even})}a \approx \Bigl(n+\tfrac12\Bigr)\pi + \frac{(n+\tfrac12)\pi}{\beta a},
\quad
k_n^{(\text{odd})}a = n\pi.
\]
Beide liegen dicht beieinander: je ein gerades und ein ungerades Niveau pro $n$, mit Abstand
\[
\Delta(k a)\approx \frac{\pi}{2} - n\pi \;+\;\varepsilon
\quad\text{bzw.}\quad
\varepsilon\sim\frac{(n+\tfrac12)\pi}{\beta a}\;\ll1.
\]
Im Grenzfall $a\to\infty$ (bei festem $\alpha$) wird $\varepsilon\to0$ und die Paare 
werden exakt entartet.

\section*{Zusammenfassung}
\begin{itemize}
  \item Für schwache Störung ($\beta a\ll1$) verschieben sich nur die geraden Zustände etwas, 
    die ungeraden bleiben bei $n\pi/a$. 
  \item Für starke Störung ($\beta a\gg1$) tritt im unteren Bereich eine fast exakte 
    Paarentartung auf: je ein gerades und ungerades Niveau pro $n$, weil der $\delta$-Impuls 
    in der Mitte den Kasten in zwei nahezu entkoppelte Hälften teilt.
  \item Physikalisch entspricht dies zwei getrennten Potentialkästen der Breite $a$,
    die nur durch eine starke Barriere gekoppelt sind; die Quanten-Tunnelung führt 
    zu schwacher Niveauteilung.
\end{itemize}

\end{document}
