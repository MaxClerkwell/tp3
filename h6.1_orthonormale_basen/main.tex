\documentclass[11pt]{scrartcl}

% Sprache und Zeichencodierung
\usepackage[utf8]{inputenc}
\usepackage[T1]{fontenc}
\usepackage[ngerman]{babel}

% Mathepakete
\usepackage{amsmath, amssymb, amsfonts}
\usepackage{enumitem} % für optionale Listeinstellungen

% Seitenränder etwas verkleinern
\usepackage[a4paper, margin=2.5cm]{geometry}

% Für fette mathematische Symbole
\usepackage{bm}

\begin{document}

\section*{Lösung zu H6.1: Orthonormale Basen}

\textbf{Gegeben:}
\[
|\xi_1\rangle \;=\;\frac12\,|1\rangle \;-\;\frac{i}{\sqrt2}\,|2\rangle,
\qquad
|\xi_2\rangle \;=\;\frac{i}{\sqrt3}\,|1\rangle \;-\;\frac{1}{\sqrt6}\,|2\rangle
\;-\;i\frac{\sqrt3}{2}\,|3\rangle.
\]

\bigskip

\section*{(a) Orthogonalisierung und Normierung}

\begin{enumerate}[label=\arabic*)]
  \item \textbf{Normen berechnen}

    \[
      \|\xi_1\|^2
      = \Bigl|\tfrac12\Bigr|^2 + \Bigl|\tfrac{i}{\sqrt2}\Bigr|^2
      = \tfrac14 + \tfrac12 = \tfrac34,
    \]
    \[
      \|\xi_2\|^2
      = \Bigl|\tfrac{i}{\sqrt3}\Bigr|^2
      + \Bigl|\tfrac{1}{\sqrt6}\Bigr|^2
      + \Bigl|\tfrac{\sqrt3}{2}\Bigr|^2
      = \tfrac13 + \tfrac16 + \tfrac34 = 2.
    \]

    Damit ergeben sich die normierten Vektoren
    \[
      |\psi_1\rangle = \frac{|\xi_1\rangle}{\sqrt{3/4}}
      = \frac{|\xi_1\rangle}{\tfrac{\sqrt3}{2}}
      = \frac{1}{\sqrt3}\,|1\rangle \;-\; i\,\sqrt{\tfrac23}\,|2\rangle,
    \]
    \[
      |\psi_2\rangle = \frac{|\xi_2\rangle}{\sqrt{2}}
      = \frac{i}{\sqrt6}\,|1\rangle 
        - \frac{1}{2\sqrt3}\,|2\rangle
        - i\,\frac{\sqrt3}{2}\,|3\rangle.
    \]

  \item \textbf{Drittes orthogonales $\boldsymbol{|\xi_3\rangle}$ bestimmen}

    Wir suchen einen Vektor
    \(
      |\xi_3\rangle = a\,|1\rangle + b\,|2\rangle + c\,|3\rangle
    \)
    mit den Orthogonalitätsbedingungen
    \[
      \langle \psi_1|\xi_3\rangle = 0,
      \qquad
      \langle \psi_2|\xi_3\rangle = 0.
    \]
    \begin{align*}
      \langle \psi_1|\xi_3\rangle &= 
        \Bigl(\tfrac{1}{\sqrt3}\langle1| + i\sqrt{\tfrac23}\langle2|\Bigr)
        (\,a|1\rangle + b|2\rangle + c|3\rangle)
      = \frac{a}{\sqrt3} + i\,\sqrt{\tfrac23}\,b = 0,\\
      \langle \psi_2|\xi_3\rangle &= 
        \Bigl(-\frac{i}{\sqrt6}\langle1| - \frac{1}{2\sqrt3}\langle2|
              + i\frac{\sqrt3}{2}\langle3|\Bigr)
        (\,a|1\rangle + b|2\rangle + c|3\rangle)\\
      &= -\frac{i\,a}{\sqrt6} - \frac{b}{2\sqrt3} + i\frac{\sqrt3}{2}\,c = 0.
    \end{align*}

    Aus der ersten Gleichung folgt
    \[
      \frac{a}{\sqrt3} + i\sqrt{\tfrac23}\,b = 0
      \;\Longrightarrow\;
      b = -\,i\,\frac{a/\sqrt3}{\sqrt{2/3}}
          = -\,i\,\frac{a}{\sqrt2}
      \;\Longrightarrow\;
      b = i\,\frac{a}{\sqrt2}\quad(\text{Vorzeichenwahl für Konsistenz}).
    \]
    Setze $b= i\,a/\sqrt2$ in die zweite Gleichung ein:
    \[
      -\frac{i\,a}{\sqrt6}
      - \frac{1}{2\sqrt3}\Bigl(i\,\frac{a}{\sqrt2}\Bigr)
      + i\frac{\sqrt3}{2}\,c = 0
      \;\Longrightarrow\;
      -\frac{i\,a}{\sqrt6}
      - i\,\frac{a}{2\sqrt6}
      + i\frac{\sqrt3}{2}\,c = 0.
    \]
    Zusammenfassen der $a$-Terme:
    \[
      i\,a\Bigl(-\tfrac{1}{\sqrt6} - \tfrac{1}{2\sqrt6}\Bigr)
      + i\frac{\sqrt3}{2}\,c = 0
      \;\Longrightarrow\;
      i\,a\Bigl(-\tfrac{3}{2\sqrt6}\Bigr)
      + i\frac{\sqrt3}{2}\,c = 0
      \;\Longrightarrow\;
      c = \frac{a\,(3/\sqrt6)}{\sqrt3}
        = a\,\frac{3}{\sqrt{18}}
        = a\,\frac{3}{3\sqrt2}
        = \frac{a}{\sqrt2}.
    \]

    Wähle nun $a=\sqrt2$ (Normierungsfreiheit), dann
    \[
      b = i\,\frac{\sqrt2}{\sqrt2} = i,
      \qquad
      c = \frac{\sqrt2}{\sqrt2} = 1.
    \]
    Somit
    \[
      |\xi_3\rangle = \sqrt2\,|1\rangle + i\,|2\rangle + 1\,|3\rangle.
    \]
    Die Norm ist
    \(\|\xi_3\| = \sqrt{2 + 1 + 1} = 2,\)
    also schließlich
    \[
      |\psi_3\rangle
      = \frac{|\xi_3\rangle}{2}
      = \frac{\sqrt2}{2}\,|1\rangle + \frac{i}{2}\,|2\rangle + \frac12\,|3\rangle.
    \]
\end{enumerate}

\bigskip

\section*{(b) Projektoren in Matrixform}

Für jeden normierten Zustand definieren wir den Projektor
\[
  P_{\psi_i} \;=\; |\psi_i\rangle\langle\psi_i|,\qquad i=1,2,3.
\]

In der Basis \(\{|1\rangle,|2\rangle,|3\rangle\}\) ergeben sich:

\[
P_{\psi_1}
= \begin{pmatrix}
\frac13 & \frac{i\sqrt2}{3} & 0 \\[4pt]
-\frac{i\sqrt2}{3} & \frac23 & 0 \\[4pt]
0 & 0 & 0
\end{pmatrix},\quad
P_{\psi_2}
= \begin{pmatrix}
\frac16 & -\frac{i\sqrt2}{12} & -\frac{1}{2\sqrt2} \\[4pt]
\frac{i\sqrt2}{12} & \frac1{12} & -\frac{i}{4} \\[4pt]
-\frac{1}{2\sqrt2} & \frac{i}{4} & \frac34
\end{pmatrix},
\]
\[
P_{\psi_3}
= \begin{pmatrix}
\frac12 & -\frac{i\sqrt2}{4} & \frac{\sqrt2}{4} \\[4pt]
\frac{i\sqrt2}{4} & \frac14 & \frac{i}{4} \\[4pt]
\frac{\sqrt2}{4} & -\frac{i}{4} & \frac14
\end{pmatrix}.
\]

\subsection*{Überprüfung}

\begin{itemize}
  \item \textbf{Hermitizität:} Jeder Projektor erfüllt
    \(P_{\psi_i}^\dagger = P_{\psi_i}\), da
    \((P_{\psi_i})_{jk} = \overline{(P_{\psi_i})_{kj}}\).
  \item \textbf{Vollständigkeit:} Es gilt
    \[
      P_{\psi_1} + P_{\psi_2} + P_{\psi_3}
      = \mathbb{I}_3,
    \]
    da die Diagonaleinträge aufsummiert \(1\) ergeben und alle Off-Diagonalen verschwinden.
\end{itemize}

\end{document}
