\documentclass[a4paper,12pt]{article}
\usepackage[utf8]{inputenc}
\usepackage{amsmath,amssymb,enumitem}
\begin{document}
\section*{Lösung H2.3 Gaußsches Wellenpaket}

Gegeben sei
\[
\Psi(x,t)
=\frac{1}{\sqrt{2\pi}}
\int_{-\infty}^{\infty}
\tilde\Psi(k)\,e^{i(kx-\omega(k)t)}\,dk,
\quad
\omega(k)=\frac{\hbar k^2}{2m},
\quad
\tilde\Psi(k)=A\,e^{-d^2(k-k_0)^2}.
\]

\begin{enumerate}[label=(\alph*)]

\item \textbf{Wahrscheinlichkeitsdichte \(P=|\Psi|^2\)}\\
Wir fassen im Exponenten quadratisch zusammen:
\[
-d^2(k-k_0)^2 + i k x - i\frac{\hbar k^2}{2m}t
=
-\Bigl(d^2 + i\frac{\hbar t}{2m}\Bigr)k^2
+ k\Bigl(2d^2k_0 + i x\Bigr)
- d^2k_0^2.
\]
Mithilfe der Gauß-Integral-Formel
\[
\int_{-\infty}^\infty e^{-\alpha k^2+\beta k}\,dk
=\sqrt{\frac{\pi}{\alpha}}\;
\exp\!\biggl(\frac{\beta^2}{4\alpha}\biggr),
\quad \Re(\alpha)>0,
\]
folgt nach Rechnung
\[
\Psi(x,t)
=A\sqrt{\frac{2d^2}{2d^2 + \tfrac{i\hbar t}{m}}}
\exp\!\Biggl[
-\frac{(x-vt)^2}{4\bigl(d^2 + \tfrac{i\hbar t}{2m}\bigr)}
+ i k_0 x - i\frac{\hbar k_0^2}{2m}t
\Biggr],
\]
mit \(v=\hbar k_0/m\). Daraus
\[
P(x,t)=|\Psi(x,t)|^2
=\frac{d}{\sqrt{2\pi\,(1+\Delta^2)}}\,
\exp\!\Bigl[-\frac{(x-vt)^2}{2\,d^2(1+\Delta^2)}\Bigr],
\quad
\Delta=\frac{\hbar t}{2 m d^2}.
\]

\item \textbf{Charakteristische Verdopplungszeit}\\
Die Breite \(\sigma_x(t)=d\sqrt{1+\Delta^2}\) verdoppelt sich für
\(\Delta=\sqrt{3}\), also
\[
t_{\rm char}
=\frac{2\,m\,d^2}{\hbar}\,\sqrt{3}.
\]
\begin{itemize}[leftmargin=*]
  \item Tennisball (\(m=5,75\cdot10^{-2}\,\mathrm{kg},\,d=6,5\cdot10^{-2}\,\mathrm m\)):
  \[
  t\approx2{,}5\times10^{23}\,\mathrm{Jahre}.
  \]
  \item Proton (\(m=1,7\cdot10^{-27}\,\mathrm{kg},\,d=10^{-15}\,\mathrm m\)):
  \[
  t\approx6\times10^{-23}\,\mathrm{s}.
  \]
\end{itemize}

\item \textbf{Verifikation mit dem Propagator}\\
Der freie Propagator
\[
K(x,t;x',0)
=\sqrt{\frac{m}{2\pi i\hbar\,t}}
\exp\!\Bigl[i\frac{m(x-x')^2}{2\hbar t}\Bigr]
\]
führt über
\(\Psi(x,t)=\int K(x,t;x',0)\,\Psi(x',0)\,dx'\)
mittels Gaußintegralen exakt auf die oben gefundene Lösung.

\item \textbf{Zeitabhängige Unschärfen}\\
Aus \(P(x,t)\) folgt
\[
\Delta x^2 = d^2\bigl(1+\Delta^2\bigr),
\quad
\Delta p^2 = \frac{\hbar^2}{4d^2},
\]
also
\[
\Delta x\,\Delta p
=\frac{\hbar}{2}\sqrt{1+\Delta^2}
\ge\frac{\hbar}{2}.
\]
\end{enumerate}

\end{document}
