\documentclass[a4paper,12pt]{article}
\usepackage[utf8]{inputenc}
\usepackage[T1]{fontenc}
\usepackage{amsmath,amssymb}
\usepackage{geometry}
\geometry{margin=2.5cm}

\begin{document}

\section*{Ausführliche Lösung zu H3.2: Zeitentwicklung eines Wellenpakets}

Wir betrachten ein freies Teilchen in einer Dimension mit dem Hamiltonoperator
\[
  \hat H = \frac{\hat p^2}{2m}\,.
\]
Seine Zustände seien zur Zeit \(t=0\) durch die Erwartungswerte
\(\langle x\rangle_0,\;\langle p\rangle_0,\;\langle xp+px\rangle_0\)
und die Unschärfen
\((\Delta x)_0,\;(\Delta p)_0\)
festgelegt. Wir wollen für \(t>0\) folgende Größen finden:
\[
  \langle x\rangle_t,\quad \langle p\rangle_t,\quad (\Delta x)_t,\quad (\Delta p)_t.
\]

\subsection*{1.\ Erwartungswerte: \(\langle x\rangle_t\) und \(\langle p\rangle_t\)}

Nach dem Ehrenfest‐Theorem gilt für jeden Operator \(\hat A\)
\[
  \frac{d}{dt}\langle \hat A\rangle
  = \frac{i}{\hbar}\,\langle [\hat H,\hat A]\rangle
  + \Bigl\langle \frac{\partial\hat A}{\partial t}\Bigr\rangle.
\]
Da \(\hat x\) und \(\hat p\) keine explizite Zeitabhängigkeit haben,
fällt der letzte Term weg.

\subsubsection*{1.1.\ Ableitung von \(\langle x\rangle\)}
Wähle \(\hat A = \hat x\). Wir benötigen
\(\bigl[\hat H,\hat x\bigr] = \bigl[\tfrac{\hat p^2}{2m},\,\hat x\bigr]\).
Nutzen wir die Identität \([AB,C]=A[B,C]+[A,C]B\):
\[
  [\hat p^2,\hat x]
  = \hat p\,[\hat p,\hat x] + [\hat p,\hat x]\,\hat p
  = \hat p\,(-i\hbar) + (-i\hbar)\,\hat p
  = -2i\hbar\,\hat p.
\]
Also
\[
  \bigl[\tfrac{\hat p^2}{2m},\hat x\bigr]
  = \frac{1}{2m}(-2i\hbar\,\hat p)
  = -\frac{i\hbar}{m}\,\hat p.
\]
Daher
\[
  \frac{d}{dt}\langle x\rangle
  = \frac{i}{\hbar}\,\Bigl\langle -\frac{i\hbar}{m}\,\hat p\Bigr\rangle
  = \frac{1}{m}\,\langle p\rangle.
\]

\subsubsection*{1.2.\ Ableitung von \(\langle p\rangle\)}
Wähle \(\hat A = \hat p\). Dann
\(\bigl[\hat H,\hat p\bigr] = \bigl[\tfrac{\hat p^2}{2m},\,\hat p\bigr]\).
Da \(\hat p\) mit sich selbst kommutiert, folgt sofort
\[
  [\hat p^2,\hat p] = 0
  \quad\Longrightarrow\quad
  \frac{d}{dt}\langle p\rangle = 0.
\]

\subsubsection*{1.3.\ Integration und Anfangsbedingungen}
Mit
\(\langle x\rangle(0)=\langle x\rangle_0\) und
\(\langle p\rangle(0)=\langle p\rangle_0\) integrieren wir:
\[
  \langle p\rangle_t = \langle p\rangle_0,
  \qquad
  \langle x\rangle_t
  = \langle x\rangle_0
    + \int_0^t \frac{\langle p\rangle_{t'}}{m}\,dt'
  = \langle x\rangle_0 + \frac{\langle p\rangle_0}{m}\,t.
\]
\[
  \boxed{
    \langle x\rangle_t
    = \langle x\rangle_0 + \frac{\langle p\rangle_0}{m}\,t,
    \quad
    \langle p\rangle_t
    = \langle p\rangle_0.
  }
\]

\subsection*{2.\ Varianzen und Unschärfen}

Wir definieren die Varianzen durch
\[
  (\Delta x)^2_t
  = \langle \hat x^2\rangle_t - \langle \hat x\rangle_t^2,
  \quad
  (\Delta p)^2_t
  = \langle \hat p^2\rangle_t - \langle \hat p\rangle_t^2.
\]
Dazu benötigen wir die Zeitentwicklung von
\(\langle \hat x^2\rangle,\;\langle \hat x\hat p+\hat p\hat x\rangle,\;\langle \hat p^2\rangle\).

\subsubsection*{2.1.\ \(\frac{d}{dt}\langle x^2\rangle\)}
Wähle \(\hat A = \hat x^2\). Dann
\[
  [\hat p^2,\hat x^2]
  = \hat p\,[\hat p,\hat x^2] + [\hat p,\hat x^2]\,\hat p.
\]
Weiter gilt
\[
  [\hat p,\hat x^2]
  = [\hat p,\hat x]\,\hat x + \hat x\,[\hat p,\hat x]
  = (-i\hbar)\,\hat x + \hat x\,(-i\hbar)
  = -2i\hbar\,\hat x.
\]
Also
\[
  [\hat p^2,\hat x^2] = \hat p\,(-2i\hbar \hat x) + (-2i\hbar \hat x)\,\hat p
  = -2i\hbar\bigl(\hat p\,\hat x + \hat x\,\hat p\bigr)
  = -2i\hbar\,(\hat x\hat p + \hat p\hat x).
\]
Daraus folgt
\[
  \frac{d}{dt}\langle x^2\rangle
  = \frac{i}{\hbar}\,\frac{1}{2m}\,
    \langle -2i\hbar \,(\hat x\hat p + \hat p\hat x)\rangle
  = \frac{1}{m}\,\langle \hat x\hat p + \hat p\hat x\rangle.
\]

\subsubsection*{2.2.\ \(\frac{d}{dt}\langle x p + p x\rangle\)}
Wähle \(\hat A = \hat x\hat p + \hat p\hat x\). Dann
\[
  [\hat p^2,\hat x\hat p + \hat p\hat x]
  = \hat p\,[\hat p,\hat x\hat p + \hat p\hat x]
    + [\hat p,\hat x\hat p + \hat p\hat x]\,\hat p.
\]
Zuerst
\[
  [\hat p,\hat x\hat p]
  = [\hat p,\hat x]\,\hat p + \hat x\,[\hat p,\hat p]
  = -i\hbar\,\hat p,
  \quad
  [\hat p,\hat p\hat x]
  = \hat p\,[\hat p,\hat x] + [\hat p,\hat p]\,\hat x
  = -i\hbar\,\hat p.
\]
Also
\[
  [\hat p,\hat x\hat p + \hat p\hat x]
  = -2i\hbar\,\hat p,
\]
und dann
\[
  [\hat p^2,\hat x\hat p + \hat p\hat x]
  = \hat p\,(-2i\hbar\,\hat p) + (-2i\hbar\,\hat p)\,\hat p
  = -4i\hbar\,\hat p^2.
\]
Daraus folgt
\[
  \frac{d}{dt}\langle x p + p x\rangle
  = \frac{i}{\hbar}\,\frac{1}{2m}\,
    \langle -4i\hbar\,\hat p^2\rangle
  = \frac{2}{m}\,\langle \hat p^2\rangle.
\]

\subsubsection*{2.3.\ \(\frac{d}{dt}\langle p^2\rangle\)}
Wähle \(\hat A=\hat p^2\). Da \(\hat p\) mit sich selbst kommutiert,
ergibt sich sofort
\[
  \frac{d}{dt}\langle p^2\rangle = 0
  \quad\Longrightarrow\quad
  \langle p^2\rangle_t = \langle p^2\rangle_0.
\]

\subsubsection*{2.4.\ Integration der Gleichungen}

Seien die Anfangswerte
\(\langle x^2\rangle_0\), \(\langle x p + p x\rangle_0\),
\(\langle p^2\rangle_0\) gegeben. Dann integrieren wir:
\[
  \langle p^2\rangle_t = \langle p^2\rangle_0,
  \quad
  \langle x p + p x\rangle_t
  = \langle x p + p x\rangle_0
    + \frac{2}{m}\,\langle p^2\rangle_0\,t,
\]
\[
  \langle x^2\rangle_t
  = \langle x^2\rangle_0
    + \int_0^t \frac{1}{m}\langle x p + p x\rangle_{t'}\,dt'
  = \langle x^2\rangle_0
    + \frac{1}{m}\,\langle x p + p x\rangle_0\,t
    + \frac{1}{m^2}\,\langle p^2\rangle_0\,t^2.
\]

\subsubsection*{2.5.\ Varianzen und Unschärfen}

Wir definieren die Kovarianz zum Zeitpunkt \(t=0\) als
\[
  \mathrm{Cov}(x,p)_0
  = \tfrac12\,\langle x p + p x\rangle_0
    - \langle x\rangle_0\,\langle p\rangle_0,
\]
und die Anfangs‐Unschärfen
\[
  (\Delta x)^2_0 = \langle x^2\rangle_0 - \langle x\rangle_0^2,
  \quad
  (\Delta p)^2_0 = \langle p^2\rangle_0 - \langle p\rangle_0^2.
\]
Damit erhält man nach etwas Umstellen
\[
  (\Delta x)^2_t
  = (\Delta x)^2_0
    + \frac{2\,\mathrm{Cov}(x,p)_0}{m}\,t
    + \frac{(\Delta p)^2_0}{m^2}\,t^2,
  \qquad
  (\Delta p)^2_t
  = (\Delta p)^2_0.
\]
Somit lauten die Unschärfen:
\[
  \boxed{
    (\Delta x)_t
    = \sqrt{\,(\Delta x)^2_0
      + \frac{2\,\mathrm{Cov}(x,p)_0}{m}\,t
      + \frac{(\Delta p)^2_0}{m^2}\,t^2\,},
    \quad
    (\Delta p)_t
    = (\Delta p)_0.
  }
\]

\bigskip

Diese Ergebnisse gelten unabhängig von der genauen Form des Wellenpakets
(also auch für nicht‐gauss’sche Zustände).

\end{document}
