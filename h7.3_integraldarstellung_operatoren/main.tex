\documentclass{scrartcl}
\usepackage[utf8]{inputenc}
\usepackage{amsmath,amssymb}
\usepackage{geometry}
\geometry{a4paper, margin=2.5cm}

\begin{document}

\section*{Lösung zu H7.3: Integraldarstellung von Operatoren}

% ------------------------------------------------------------
% Dieser Abschnitt beschreibt die Aufgabenstellung und die Vorgehensweise.
% ------------------------------------------------------------
\subsection*{Aufgabenstellung}
Gegeben sei ein Operator $\hat{A}$ im Ortsraum durch
\[
\hat{A}\,\lvert \boldsymbol{r} \rangle \;:=\; \frac{1}{r}\,e^{-m r}\,\lvert \boldsymbol{r} \rangle,\qquad m>0.
\]
Wir sollen den Kern $A(\boldsymbol{p},\boldsymbol{p}^{\prime})$ in der Impulsraumdarstellung berechnen, das heißt:
\[
A(\boldsymbol{p},\boldsymbol{p}^{\prime})
\;=\;
\langle \boldsymbol{p}\rvert\,\hat{A}\,\lvert \boldsymbol{p}^{\prime}\rangle,
\]
und physikalisch deuten, was im Grenzfall $m \to 0$ passiert.

% ------------------------------------------------------------
% Hinweis auf Fourier-Transformation und Darstellung des Operators.
% ------------------------------------------------------------
\subsection*{1. Darstellung des Ortsraumkerns von $\hat{A}$}
\begin{itemize}
  \item Wir beginnen damit, den \emph{Kern im Ortsraum} $A(\boldsymbol{r},\boldsymbol{r}^{\prime})$ zu bestimmen. Per Definition gilt:
  \[
    (\hat{A}\,\Psi)(\boldsymbol{r})
    \;=\; \langle \boldsymbol{r}\rvert \hat{A} \lvert \Psi \rangle
    \;=\; \frac{1}{r}\,e^{-m r}\,\Psi(\boldsymbol{r}).
  \]
  \item Daraus folgt unmittelbar, dass
  \[
    A(\boldsymbol{r},\boldsymbol{r}^{\prime})
    \;=\;
    \frac{1}{r}\,e^{-m r}\;\delta^{(3)}(\boldsymbol{r} - \boldsymbol{r}^{\prime}),
    \quad\text{mit }r = |\boldsymbol{r}|.
  \]
  \item \textbf{Zwischenkommentar:} 
    Der Operator wirkt lokal im Ortsraum und multipliziert $\Psi(\boldsymbol{r})$ mit der Funktion $(1/r)e^{-m r}$, daher der Dirac-Delta-Faktor.
\end{itemize}

% ------------------------------------------------------------
% Abschnitt zur Umrechnung in den Impulsraum über Fourier-Transformation.
% ------------------------------------------------------------
\subsection*{2. Übergang zum Impulsraum}
\subsubsection*{2.1. Fourier-Darstellung der Impuls- und Ortszustände}
\begin{itemize}
  \item Man wendet die Fourier-Transformation an, um die Impulsraum-Matrixelemente zu berechnen:
  \[
    \langle \boldsymbol{p} \rvert \boldsymbol{r} \rangle
    \;=\;
    \frac{1}{(2\pi\hbar)^{3/2}}
    \exp\!\Bigl(\tfrac{i}{\hbar}\,\boldsymbol{p}\cdot\boldsymbol{r}\Bigr),
    \qquad
    \langle \boldsymbol{r} \rvert \boldsymbol{p}^{\prime} \rangle
    \;=\;
    \frac{1}{(2\pi\hbar)^{3/2}}
    \exp\!\Bigl(-\tfrac{i}{\hbar}\,\boldsymbol{p}^{\prime}\cdot\boldsymbol{r}\Bigr).
  \]
  \item \textbf{Zwischenkommentar:} 
    In der üblichen Konvention sind $\langle \boldsymbol{r}|\boldsymbol{p}\rangle = (2\pi\hbar)^{-3/2} e^{+\,i\boldsymbol{p}\cdot\boldsymbol{r}/\hbar}$ und $\langle \boldsymbol{p}|\boldsymbol{r}\rangle = (2\pi\hbar)^{-3/2} e^{-\,i\boldsymbol{p}\cdot\boldsymbol{r}/\hbar}$.
\end{itemize}

\subsubsection*{2.2. Ausdruck für $A(\boldsymbol{p},\boldsymbol{p}^{\prime})$}
\begin{align*}
  A(\boldsymbol{p},\boldsymbol{p}^{\prime})
  \;=\; 
  \langle \boldsymbol{p} \rvert \hat{A} \lvert \boldsymbol{p}^{\prime} \rangle
  &=\;
  \int d^{3}r \int d^{3}r^{\prime}\,
  \langle \boldsymbol{p}\rvert\boldsymbol{r}\rangle\,
  A(\boldsymbol{r},\boldsymbol{r}^{\prime})\,
  \langle \boldsymbol{r}^{\prime}\rvert\boldsymbol{p}^{\prime}\rangle
  \\[1ex]
  &=\;
  \int d^{3}r \int d^{3}r^{\prime}\,
  \frac{e^{-\,\tfrac{i}{\hbar}\,\boldsymbol{p}\cdot\boldsymbol{r}}}{(2\pi\hbar)^{3/2}}\,
  \Bigl[\frac{1}{r}e^{-m r}\,\delta^{(3)}(\boldsymbol{r}-\boldsymbol{r}^{\prime})\Bigr]\,
  \frac{e^{+\,\tfrac{i}{\hbar}\,\boldsymbol{p}^{\prime}\cdot\boldsymbol{r}^{\prime}}}{(2\pi\hbar)^{3/2}}
  \\[1ex]
  &=\;
  \frac{1}{(2\pi\hbar)^{3}}
  \int d^{3}r \; \frac{e^{-\,\tfrac{i}{\hbar}\,\boldsymbol{p}\cdot\boldsymbol{r}}}{\;}\;
  \frac{e^{+\,\tfrac{i}{\hbar}\,\boldsymbol{p}^{\prime}\cdot\boldsymbol{r}}}{\;}\;
  \frac{e^{-m r}}{r}.
\end{align*}
\begin{itemize}
  \item \textbf{Zwischenkommentar:} 
    Aus der Delta-Funktion ergibt sich $\boldsymbol{r}^{\prime} = \boldsymbol{r}$. Die beiden Exponentialfaktoren verschmelzen zu $e^{-\,\tfrac{i}{\hbar}(\boldsymbol{p}-\boldsymbol{p}^{\prime})\cdot\boldsymbol{r}}$. Wir definieren den Impulsdifferenzvektor
    \[
      \boldsymbol{q} \;:=\; \boldsymbol{p} - \boldsymbol{p}^{\prime}.
    \]
  \item Damit vereinfacht sich
  \[
    A(\boldsymbol{p},\boldsymbol{p}^{\prime})
    \;=\;
    \frac{1}{(2\pi\hbar)^{3}}\,
    \int d^{3}r\,\frac{e^{-m r}}{r}
    \exp\!\Bigl[-\,\tfrac{i}{\hbar}\,\boldsymbol{q}\cdot\boldsymbol{r}\Bigr].
  \]
\end{itemize}

% ------------------------------------------------------------
% Abschnitt zur Auswertung des Integrals in Kugelkoordinaten.
% ------------------------------------------------------------
\subsection*{3. Auswertung des räumlichen Integrals}
\subsubsection*{3.1. Umwandlung in Kugelkoordinaten}
\begin{itemize}
  \item Wir wählen Kugelkoordinaten: 
    \[
      d^{3}r = r^{2}\,\sin\theta\,dr\,d\theta\,d\varphi,\quad
      r = |\boldsymbol{r}|,\quad
      \boldsymbol{q}\cdot\boldsymbol{r} = q\,r\,\cos\theta,
      \quad q = |\boldsymbol{q}\,|.
    \]
  \item Das Integral zerfällt in einen Radial- und einen Winkelanteil:
  \[
    \int d^{3}r\,\frac{e^{-m r}}{r}e^{-\,\tfrac{i}{\hbar}\,\boldsymbol{q}\cdot\boldsymbol{r}}
    \;=\;
    \int_{0}^{\infty} dr\;\int_{0}^{\pi} d\theta\;\int_{0}^{2\pi} d\varphi\;
    \,\Bigl[r^{2}\sin\theta\,\frac{e^{-m r}}{r}\Bigr]\,
    \exp\!\Bigl[-\,\tfrac{i}{\hbar}q\,r\,\cos\theta\Bigr].
  \]
  \item Vereinfachung: $r^{2}\,(1/r) = r$. Somit
  \[
    \int d^{3}r\,\frac{e^{-m r}}{r}e^{-\,\tfrac{i}{\hbar}\,\boldsymbol{q}\cdot\boldsymbol{r}}
    =
    \int_{0}^{\infty} dr\,r\,e^{-m r}
    \Biggl[\int_{0}^{\pi} d\theta\,\sin\theta\,
    e^{-\,\tfrac{i}{\hbar}q\,r\,\cos\theta}\,\Biggr]
    \Bigl[\int_{0}^{2\pi} d\varphi\Bigr].
  \]
  \item Der $\varphi$-Anteil liefert 
  \[
    \int_{0}^{2\pi} d\varphi = 2\pi.
  \]
\end{itemize}

\subsubsection*{3.2. Berechnung des Winkelintegrals über $\theta$}
\begin{itemize}
  \item Wir betrachten das $\theta$-Integral:
  \[
    I_{\theta}
    = 
    \int_{0}^{\pi}\!\sin\theta\,
    \exp\!\bigl[-\,\tfrac{i}{\hbar}q\,r\,\cos\theta\bigr]\;d\theta.
  \]
  \item Substitution: $u = \cos\theta \;\Rightarrow\; du = -\sin\theta\,d\theta$. Wenn $\theta$ von $0$ bis $\pi$ läuft, dann läuft $u$ von $1$ bis $-1$. Also
  \[
    I_{\theta}
    = 
    \int_{u=1}^{u=-1}
    \exp\!\bigl[-\,\tfrac{i}{\hbar}q\,r\,u\bigr]\,(-du)
    = 
    \int_{-1}^{1} \exp\!\bigl[-\,\tfrac{i}{\hbar}q\,r\,u\bigr]\,du.
  \]
  \item Auswertung dieses Integrals:
  \[
    I_{\theta}
    = 
    \Bigl[\;-\frac{\hbar}{i\,q\,r}\,
    \exp\!\bigl(-\,\tfrac{i}{\hbar}q\,r\,u\bigr)\Bigr]_{u=-1}^{\,u=1}
    \;=\;
    \frac{\hbar}{i\,q\,r}\,
    \Bigl[e^{-\,\tfrac{i}{\hbar}q\,r} \;-\; e^{+\,\tfrac{i}{\hbar}q\,r}\Bigr].
  \]
  \item Mit der Identität $e^{-ix}-e^{+ix} = -\,2\,i\,\sin(x)$ folgt
  \[
    I_{\theta}
    = 
    \frac{\hbar}{i\,q\,r}\,\bigl(-\,2\,i\bigr)\,
    \sin\!\bigl(\tfrac{q\,r}{\hbar}\bigr)
    = 
    \frac{2\,\hbar}{q\,r}\,
    \sin\!\bigl(\tfrac{q\,r}{\hbar}\bigr).
  \]
  \item Damit ist das volle Winkelintegral
  \[
    \int_{0}^{\pi}d\theta\,\sin\theta\,
    e^{-\,\tfrac{i}{\hbar}q\,r\,\cos\theta}
    \;=\;
    \frac{2\,\hbar}{q\,r}\,
    \sin\!\bigl(\tfrac{q\,r}{\hbar}\bigr).
  \]
\end{itemize}

\subsubsection*{3.3. Zusammensetzen der Teile}
\begin{itemize}
  \item Zusammenfassung: 
  \[
    \int d^{3}r\,\frac{e^{-m r}}{r}\,e^{-\,\tfrac{i}{\hbar}\,\boldsymbol{q}\cdot\boldsymbol{r}}
    = 
    2\pi \int_{0}^{\infty} dr \; r \,e^{-m r}\;
    \Bigl[\,\underbrace{\frac{2\,\hbar}{q\,r}\,\sin\bigl(\tfrac{q\,r}{\hbar}\bigr)}_{\substack{\text{Winkelintegral}\\I_{\theta}}}\Bigr].
  \]
  \item Vereinfachen:
  \[
    = 
    2\pi \,\frac{2\,\hbar}{q}\,
    \int_{0}^{\infty} dr \; e^{-m r}\,\sin\!\bigl(\tfrac{q\,r}{\hbar}\bigr).
    \quad % r kürzt sich weg
  \]
  \item Definiere $k = \frac{q}{\hbar}$. Dann ist
  \[
    \int_{0}^{\infty} dr \; e^{-m r}\,\sin(k r)
    \;=\;
    \frac{k}{\,m^{2} + k^{2}\,},
    \quad\text{bekannte Standardformel.}
  \]
  \item Setzen wir $k = \tfrac{q}{\hbar}$ ein:
  \[
    \int_{0}^{\infty} dr \; e^{-m r}\,\sin\!\Bigl(\tfrac{q\,r}{\hbar}\Bigr)
    \;=\;
    \frac{\tfrac{q}{\hbar}}{\,m^{2} + \bigl(\tfrac{q}{\hbar}\bigr)^{2}\!}
    \;=\;
    \frac{q/\hbar}{\,m^{2} + q^{2}/\hbar^{2}\!}
    \;=\;
    \frac{q}{\,\hbar\,\bigl(m^{2} + q^{2}/\hbar^{2}\bigr)\!}.
  \]
  \item Damit ergibt sich
  \[
    \int d^{3}r\,\frac{e^{-m r}}{r}\,e^{-\,\tfrac{i}{\hbar}\,\boldsymbol{q}\cdot\boldsymbol{r}}
    = 
    2\pi \,\frac{2\,\hbar}{q}
    \;\Bigl[\frac{q}{\,\hbar\,(m^{2} + q^{2}/\hbar^{2})\,}\Bigr]
    = 
    4\pi\,\frac{1}{\,m^{2} + \tfrac{q^{2}}{\hbar^{2}}\,}.
  \]
\end{itemize}

% ------------------------------------------------------------
% Endgültiges Ergebnis für den Impulskern A(p,p').
% ------------------------------------------------------------
\subsection*{4. Endergebnis für $A(\boldsymbol{p},\boldsymbol{p}^{\prime})$}
\begin{itemize}
  \item Wir hatten:
  \[
    A(\boldsymbol{p},\boldsymbol{p}^{\prime})
    \;=\;
    \frac{1}{(2\pi\hbar)^{3}}\,
    \int d^{3}r\,\frac{e^{-m r}}{r}
    \exp\!\Bigl[-\,\tfrac{i}{\hbar}(\boldsymbol{p}-\boldsymbol{p}^{\prime})\cdot\boldsymbol{r}\Bigr].
  \]
  \item Das räumliche Integral haben wir ausgewertet und gefunden:
  \[
    \int d^{3}r\,\frac{e^{-m r}}{r}
    e^{-\,\tfrac{i}{\hbar}\,\boldsymbol{q}\cdot\boldsymbol{r}}
    \;=\;
    4\pi\,
    \frac{1}{\,m^{2} + \tfrac{q^{2}}{\hbar^{2}}\!},
    \quad q = \lvert \boldsymbol{p} - \boldsymbol{p}^{\prime}\rvert.
  \]
  \item Damit:
  \[
    A(\boldsymbol{p},\boldsymbol{p}^{\prime})
    \;=\;
    \frac{1}{(2\pi\hbar)^{3}}\;
    4\pi\;\frac{1}{\,m^{2} + \tfrac{(\lvert \boldsymbol{p}-\boldsymbol{p}^{\prime}\rvert)^{2}}{\hbar^{2}}\!}
    \;=\;
    \frac{4\pi}{(2\pi\hbar)^{3}}\;
    \frac{1}{\,m^{2} + \tfrac{(\boldsymbol{p}-\boldsymbol{p}^{\prime})^{2}}{\hbar^{2}}\!}.
  \]
  \item Wir können den Nenner wie folgt umschreiben:
  \[
    m^{2} + \frac{(\boldsymbol{p}-\boldsymbol{p}^{\prime})^{2}}{\hbar^{2}}
    \;=\;
    \frac{(\boldsymbol{p}-\boldsymbol{p}^{\prime})^{2} + m^{2}\,\hbar^{2}}{\hbar^{2}}.
  \]
  \item Deshalb folgt:
  \[
    A(\boldsymbol{p},\boldsymbol{p}^{\prime})
    \;=\;
    \frac{4\pi}{(2\pi\hbar)^{3}}\;
    \frac{\hbar^{2}}{\,(\boldsymbol{p}-\boldsymbol{p}^{\prime})^{2} + m^{2}\,\hbar^{2}\,}
    \;=\;
    \frac{4\pi\,\hbar^{2}}{\,8\pi^{3}\,\hbar^{3}\,\bigl[(\boldsymbol{p}-\boldsymbol{p}^{\prime})^{2} + m^{2}\,\hbar^{2}\bigr]\,}.
  \]
  \item Kürzen: $4\pi/8\pi^{3} = 1/(2\pi^{2})$, und $\hbar^{2}/\hbar^{3} = 1/\hbar$. Daher
  \[
    \boxed{
      A(\boldsymbol{p},\boldsymbol{p}^{\prime})
      \;=\;
      \frac{1}{\,2\pi^{2}\,\hbar\,}\;
      \frac{1}{\,(\boldsymbol{p} - \boldsymbol{p}^{\prime})^{2} + (m\,\hbar)^{2}\,}.
    }
  \]
  \item \textbf{Zwischenkommentar:} 
    Dieses Ergebnis entspricht dem gut bekannten Impulsraumkern des Yukawa-Potentials $(e^{-m r}/r)$.
\end{itemize}

% ------------------------------------------------------------
% Physikalische Bedeutung des Grenzfalls m -> 0.
% ------------------------------------------------------------
\subsection*{5. Physikalische Bedeutung des Grenzfalls $m \to 0$}
\begin{itemize}
  \item Wenn wir $m \to 0$ setzen, vereinfacht sich der Kern zu
  \[
    A(\boldsymbol{p},\boldsymbol{p}^{\prime}) \;\xrightarrow{m\to 0}\;
    \frac{1}{\,2\pi^{2}\,\hbar\,}\;
    \frac{1}{\,(\boldsymbol{p} - \boldsymbol{p}^{\prime})^{2}\,}.
  \]
  \item \textbf{Zwischenkommentar:} 
    Das entspricht dem Impulsraumkern des Coulomb-Potentials $\frac{1}{r}$. Denn im Grenzfall $m=0$ hat $e^{-m r}/r \to 1/r$.
  \item \emph{Physikalische Interpretation:} 
  \begin{itemize}
    \item Der Parameter $m$ steht in der Yukawa-Wechselwirkung für die inverse Reichweite $\lambda = 1/m$. 
    \item Für $m>0$ ist das Potential $V(r) = (1/r)\,e^{-m r}$ kurzreichweitig, typisch für Austauschmassive Teilchen (z.\,B. im Kernbereich: Pionenaustausch).
    \item Im Grenzfall $m \to 0$ wird die Reichweite unendlich groß, sodass das Potential zu $V(r) = 1/r$ wird: das klassische Coulomb-Potential. 
    \item Folglich beschreibt $\hat{A}$ bei $m=0$ den Operator, der im Ortsraum mit $1/r$ multipliziert. Sein Impulsraumkern ist dann das bekannte $1/(\boldsymbol{q}^{2})$ (bis Konstanten).
  \end{itemize}
  \item \textbf{Zwischenkommentar:} 
    In der Quantenfeldtheorie würde man $m=0$ dem Fall masseloser Austauschteilchen (z.\,B. Photonen) zuordnen, die eine Coulomb-Wechselwirkung vermitteln.
\end{itemize}

\end{document}
