\documentclass[a4paper,12pt]{scrartcl}
\usepackage[utf8]{inputenc}
\usepackage{amsmath}
\usepackage{amssymb}
\usepackage{physics}


\begin{document}

\section*{Lösung zu H6.2: Hellmann-Feynman-Theorem}

Sei $\hat{H}_\lambda$ ein Hamiltonoperator, der von einem reellen Parameter $\lambda$ abhängt. Die Eigenwertgleichung lautet
\begin{equation}
    \hat{H}_\lambda \ket{\Psi_\lambda} = E_\lambda \ket{\Psi_\lambda},
\end{equation}
wobei $\ket{\Psi_\lambda}$ ein normierter Eigenzustand ist, also $\braket{\Psi_\lambda}{\Psi_\lambda} = 1$.

Zeigen Sie, dass für den Eigenwert $E_\lambda$ die folgende Identität gilt:
\begin{equation}
    \frac{\partial E_\lambda}{\partial \lambda} = \mel{\Psi_\lambda}{\frac{\partial \hat{H}_\lambda}{\partial \lambda}}{\Psi_\lambda}.
\end{equation}

\section*{Lösung}

Wir beginnen mit der Ableitung des Eigenwerts $E_\lambda$ nach dem Parameter $\lambda$:
\begin{equation}
    E_\lambda = \bra{\Psi_\lambda} \hat{H}_\lambda \ket{\Psi_\lambda}.
\end{equation}

Dies ist gültig, weil $\hat{H}_\lambda \ket{\Psi_\lambda} = E_\lambda \ket{\Psi_\lambda}$ und $\braket{\Psi_\lambda}{\Psi_\lambda} = 1$.

Nun leiten wir beide Seiten dieser Gleichung nach $\lambda$ ab:
\begin{equation}
    \frac{\partial E_\lambda}{\partial \lambda} = \frac{\partial}{\partial \lambda} \bra{\Psi_\lambda} \hat{H}_\lambda \ket{\Psi_\lambda}.
\end{equation}

Wir wenden die Produktregel (Kettenregel) auf die Ableitung eines Skalarprodukts mit Operator an:
\begin{align}
    \frac{\partial E_\lambda}{\partial \lambda}
    &= \left( \frac{\partial}{\partial \lambda} \bra{\Psi_\lambda} \right) \hat{H}_\lambda \ket{\Psi_\lambda}
    + \bra{\Psi_\lambda} \frac{\partial \hat{H}_\lambda}{\partial \lambda} \ket{\Psi_\lambda}
    + \bra{\Psi_\lambda} \hat{H}_\lambda \left( \frac{\partial}{\partial \lambda} \ket{\Psi_\lambda} \right).
\end{align}

Bezeichne nun:
\begin{equation}
    \ket{\dot{\Psi}_\lambda} = \frac{\partial}{\partial \lambda} \ket{\Psi_\lambda}, \qquad
    \bra{\dot{\Psi}_\lambda} = \frac{\partial}{\partial \lambda} \bra{\Psi_\lambda}.
\end{equation}

Dann können wir schreiben:
\begin{align}
    \frac{\partial E_\lambda}{\partial \lambda}
    &= \bra{\dot{\Psi}_\lambda} \hat{H}_\lambda \ket{\Psi_\lambda}
    + \bra{\Psi_\lambda} \frac{\partial \hat{H}_\lambda}{\partial \lambda} \ket{\Psi_\lambda}
    + \bra{\Psi_\lambda} \hat{H}_\lambda \ket{\dot{\Psi}_\lambda}.
\end{align}

Da $\hat{H}_\lambda$ ein hermitescher Operator ist und $\hat{H}_\lambda \ket{\Psi_\lambda} = E_\lambda \ket{\Psi_\lambda}$, gilt:
\begin{equation}
    \bra{\dot{\Psi}_\lambda} \hat{H}_\lambda \ket{\Psi_\lambda} = E_\lambda \braket{\dot{\Psi}_\lambda}{\Psi_\lambda},
\end{equation}
\begin{equation}
    \bra{\Psi_\lambda} \hat{H}_\lambda \ket{\dot{\Psi}_\lambda} = E_\lambda \braket{\Psi_\lambda}{\dot{\Psi}_\lambda}.
\end{equation}

Wegen $\braket{\Psi_\lambda}{\Psi_\lambda} = 1$ gilt:
\begin{equation}
    \frac{\partial}{\partial \lambda} \braket{\Psi_\lambda}{\Psi_\lambda} = 0,
\end{equation}
\begin{equation}
    \braket{\dot{\Psi}_\lambda}{\Psi_\lambda} + \braket{\Psi_\lambda}{\dot{\Psi}_\lambda} = 0.
\end{equation}

Das bedeutet, dass
\begin{equation}
    \braket{\dot{\Psi}_\lambda}{\Psi_\lambda} = -\braket{\Psi_\lambda}{\dot{\Psi}_\lambda}.
\end{equation}

Somit ist
\begin{equation}
    \bra{\dot{\Psi}_\lambda} \hat{H}_\lambda \ket{\Psi_\lambda} + \bra{\Psi_\lambda} \hat{H}_\lambda \ket{\dot{\Psi}_\lambda}
    = E_\lambda \left( \braket{\dot{\Psi}_\lambda}{\Psi_\lambda} + \braket{\Psi_\lambda}{\dot{\Psi}_\lambda} \right) = 0.
\end{equation}

Damit folgt schließlich:
\begin{equation}
    \frac{\partial E_\lambda}{\partial \lambda}
    = \bra{\Psi_\lambda} \frac{\partial \hat{H}_\lambda}{\partial \lambda} \ket{\Psi_\lambda}.
\end{equation}

\section*{Schlussfolgerung}

Dies ist die Aussage des \textbf{Hellmann-Feynman-Theorems}. Es erlaubt die Berechnung der Ableitung des Eigenwerts $E_\lambda$ allein aus der Ableitung des Operators $\hat{H}_\lambda$ — ohne explizite Kenntnis der Ableitung des Zustandsvektors $\ket{\Psi_\lambda}$. Dies ist besonders nützlich in quantenmechanischen Rechnungen, z.\,B.\ bei der Bestimmung von Kraft- oder Energieänderungen in Molekülsystemen.

\end{document}
