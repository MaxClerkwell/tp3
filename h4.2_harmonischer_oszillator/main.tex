\documentclass{article}
\usepackage[utf8]{inputenc}
\usepackage[T1]{fontenc}
\usepackage{amsmath,amssymb}


\begin{document}

\section{Lösung zu H4.2: Harmonischer Oszillator}
\subsection*{Erwartungswerte von $\hat x$ und $\hat p$}
Wir schreiben die Orts- und Impulsoperatoren in Leiteroperatoren um:
\[
  \hat x = \sqrt{\frac{\hbar}{2m\omega}}\,(a + a^\dagger), 
  \qquad
  \hat p = i\sqrt{\frac{\hbar m\omega}{2}}\,(a^\dagger - a).
\]
Für den Zustand $|n\rangle$ gelten die Wirkungen
\[
  a\,|n\rangle = \sqrt{n}\,|n-1\rangle,
  \quad
  a^\dagger\,|n\rangle = \sqrt{n+1}\,|n+1\rangle.
\]
Daher sind die Diagonalelemente
\[
  \langle n|a|n\rangle = 0,
  \quad
  \langle n|a^\dagger|n\rangle = 0,
\]
weil $a$ bzw.\ $a^\dagger$ die Quantenzahl $n$ ändern. Somit folgt:
\[
  \langle \hat x \rangle_n 
  = \sqrt{\frac{\hbar}{2m\omega}}\bigl(\langle n|a|n\rangle + \langle n|a^\dagger|n\rangle\bigr)
  = 0,
\]
\[
  \langle \hat p \rangle_n
  = i\sqrt{\frac{\hbar m\omega}{2}}\bigl(\langle n|a^\dagger|n\rangle - \langle n|a|n\rangle\bigr)
  = 0.
\]

\subsection*{Berechnung von $\langle \hat x^2 \rangle_n$}
Wir betrachten das Quadrat des Ortsoperators:
\[
  \hat x^2 
  = \frac{\hbar}{2m\omega}\,(a + a^\dagger)^2
  = \frac{\hbar}{2m\omega}\bigl(a^2 + a\,a^\dagger + a^\dagger\,a + a^{\dagger2}\bigr).
\]
Damit ist
\[
  \langle \hat x^2 \rangle_n
  = \frac{\hbar}{2m\omega}\Bigl(\langle n|a^2|n\rangle 
    + \langle n|a\,a^\dagger|n\rangle 
    + \langle n|a^\dagger\,a|n\rangle 
    + \langle n|a^{\dagger2}|n\rangle\Bigr).
\]
Nun gilt:
\[
  \langle n|a^2|n\rangle = 0, 
  \quad
  \langle n|a^{\dagger2}|n\rangle = 0,
\]
da $a^2|n\rangle\propto|n-2\rangle$ und $a^{\dagger2}|n\rangle\propto|n+2\rangle$ orthogonal zu $|n\rangle$.  
Mit der Vertauschungsrelation $[a,a^\dagger]=1$, also $a\,a^\dagger = a^\dagger\,a + 1$, folgt:
\[
  \langle n|a\,a^\dagger|n\rangle
  = \langle n|(a^\dagger\,a + 1)|n\rangle
  = \langle n|a^\dagger\,a|n\rangle + 1
  = n + 1,
\]
\[
  \langle n|a^\dagger\,a|n\rangle = n.
\]
Also
\[
  \langle \hat x^2 \rangle_n
  = \frac{\hbar}{2m\omega}\bigl((n+1) + n\bigr)
  = \frac{\hbar}{2m\omega}(2n+1).
\]

\subsection*{Berechnung von $\langle \hat p^2 \rangle_n$}
Analog zum Ortsoperator:
\[
  \hat p^2
  = -\frac{\hbar m\omega}{2}\,(a^\dagger - a)^2
  = \frac{\hbar m\omega}{2}\bigl(a\,a^\dagger + a^\dagger\,a - a^2 - a^{\dagger2}\bigr).
\]
Die Terme mit $a^2$ und $a^{\dagger2}$ verschwinden erneut, somit:
\[
  \langle \hat p^2 \rangle_n
  = \frac{\hbar m\omega}{2}\bigl((n+1) + n\bigr)
  = \frac{\hbar m\omega}{2}(2n+1).
\]

\subsection*{Standardabweichungen $(\Delta x)_n$ und $(\Delta p)_n$}
Die Varianzen sind:
\[
  (\Delta x)_n^2 
  = \langle \hat x^2 \rangle_n - \langle \hat x \rangle_n^2
  = \frac{\hbar}{2m\omega}(2n+1),
\]
\[
  (\Delta p)_n^2
  = \langle \hat p^2 \rangle_n - \langle \hat p \rangle_n^2
  = \frac{\hbar m\omega}{2}(2n+1).
\]
Daraus:
\[
  (\Delta x)_n = \sqrt{\frac{\hbar}{2m\omega}(2n+1)},
  \quad
  (\Delta p)_n = \sqrt{\frac{\hbar m\omega}{2}(2n+1)}.
\]

\subsection*{Heisenbergsche Unschärferelation}
\[
  (\Delta x)_n\,(\Delta p)_n
  = \sqrt{\frac{\hbar}{2m\omega}(2n+1)}
    \,\sqrt{\frac{\hbar m\omega}{2}(2n+1)}
  = (2n+1)\,\frac{\hbar}{2}
  \;\ge\;\frac{\hbar}{2}.
\]
Für den Grundzustand $n=0$ gilt die Gleichheit:
\[
  (\Delta x)_0\,(\Delta p)_0 = \frac{\hbar}{2}.
\]

\end{document}
