\documentclass[a4paper,11pt]{article}
\usepackage[utf8]{inputenc}       % UTF-8 Zeichencodierung
\usepackage[T1]{fontenc}          % T1-Schriftkodierung für korrekte Anführungszeichen
\usepackage{amsmath,amssymb}      % Mathe-Erweiterungen
\usepackage{physics}              % \vb*, \dv, \pdv, etc.
\usepackage{hyperref}             % Verlinkungen

\begin{document}

\section*{Lösung zu H2.1 Kugelwellen}

Wir betrachten die zeitabhängige Schrödinger-Gleichung für ein freies Teilchen der Masse \(m\) in drei Dimensionen,
\[
i\hbar\,\frac{\partial}{\partial t}\Psi(\vb*{x},t)
=
-\frac{\hbar^2}{2m}\,\nabla^2\Psi(\vb*{x},t)\,.
\]
Im Folgenden zeigen wir, dass sich in \(\vb*{x}\neq \vb*{0}\) spezielle Lösungen in Form von Kugelwellen
\[
\Psi(\vb*{x},t)
= A\,\frac{1}{r}\,\exp\bigl(i\,(k\,r - \omega\,t)\bigr),
\quad
r = \lvert \vb*{x}\rvert,
\quad
\omega = \frac{\hbar k^2}{2m},
\]
finden lassen.

\subsection*{(a) Nachweis der Kugelwellen-Lösung}

\paragraph{Ansatz.}
Wir setzen
\[
\Psi(\vb*{x},t)
= A\,\frac{e^{i(kr - \omega t)}}{r},
\]
mit Konstanten \(A\), \(k\) und \(\omega\).

\paragraph{Laplacian in Kugelkoordinaten.}
Für radialsymmetrische Funktionen \(f(r)\) gilt in Kugelkoordinaten
\[
\nabla^2 f(r)
= \frac{1}{r^2}\,\frac{\mathrm{d}}{\mathrm{d}r}
\Bigl(r^2\,\frac{\mathrm{d}f}{\mathrm{d}r}\Bigr).
\]
Man zeigt (durch Einsetzen von \(f(r) = e^{ikr}/r\)), dass
\[
\nabla^2\Bigl(\frac{e^{ikr}}{r}\Bigr)
= -k^2\,\frac{e^{ikr}}{r}.
\]
Damit ergibt sich auf der rechten Seite der Schrödinger-Gleichung
\[
-\frac{\hbar^2}{2m}\,\nabla^2\Psi
= \frac{\hbar^2 k^2}{2m}\,\Psi.
\]
Die linke Seite liefert
\[
i\hbar\,\frac{\partial\Psi}{\partial t}
= \hbar\omega\,\Psi.
\]
Beide Seiten stimmen überein, wenn
\[
\hbar\omega = \frac{\hbar^2 k^2}{2m}
\;\;\Longrightarrow\;\;
\omega = \frac{\hbar k^2}{2m},
\]
was zu zeigen war.

\subsection*{(b) Wahrscheinlichkeitsstromdichte und Kontinuitätsgleichung}

\paragraph{Definition des Wahrscheinlichkeitsstroms.}
Für eine Wellenfunktion \(\Psi\) ist der Strom
\[
\vb*{j}(\vb*{x},t)
= \frac{\hbar}{2mi}\,\bigl(\Psi^*\,\nabla\Psi - \Psi\,\nabla\Psi^*\bigr).
\]
Da \(\Psi\) nur von \(r\) abhängt und \(\nabla \Psi = \hat{\vb*{r}}\,\frac{\mathrm{d}\Psi}{\mathrm{d}r}\),
ergibt sich nach etwas Rechnung
\[
\vb*{j}(r)
= \hat{\vb*{r}}\,\frac{\hbar k}{m}\,\frac{\lvert A\rvert^2}{r^2}.
\]

\paragraph{Kontinuitätsgleichung.}
Die Kontinuitätsgleichung fordert
\[
\frac{\partial \rho}{\partial t} + \nabla\!\cdot\!\vb*{j} = 0,
\qquad
\rho = \lvert \Psi\rvert^2 = \frac{\lvert A\rvert^2}{r^2}.
\]
Offensichtlich ist \(\rho\) zeitunabhängig, also \(\partial_t\rho=0\), und
\[
\nabla\!\cdot\!\vb*{j}
= \frac{1}{r^2}\,\frac{\mathrm{d}}{\mathrm{d}r}\bigl(r^2 j_r(r)\bigr)
= 0,
\]
sodass die Kontinuitätsgleichung erfüllt ist.

\subsection*{(c) Wahrscheinlichkeitsfluss durch geschlossene Oberflächen}

Der Fluss durch eine geschlossene Fläche \(S\) ergibt sich zu
\[
\Phi_S
= \oint_S \vb*{j}\cdot \mathrm{d}\vb*{S}
= \frac{\hbar k}{m}\,\lvert A\rvert^2
\;\oint_S \mathrm{d}\Omega.
\]
\begin{itemize}
  \item[(i)] Schließt \(S\) den Ursprung ein (\(0\in V_S\)), so ist
  \(\oint_S \mathrm{d}\Omega = 4\pi\) und
  \(\Phi_S = \frac{4\pi\hbar k}{m}\,\lvert A\rvert^2\).
  \item[(ii)] Ohne Ursprung im Inneren gilt \(\nabla\!\cdot\!\vb*{j}=0\) für \(r\neq0\)
  und nach Gauß’ Gesetz \(\Phi_S=0\).
\end{itemize}

Physikalisch wird der Fluss aus einem punktförmigen Zentrum ausgesandt, während
außerhalb kein Nettofluss entsteht.

\end{document}
