\documentclass[a4paper,11pt]{article}
\usepackage[utf8]{inputenc}
\usepackage{amsmath}
\usepackage{amssymb}
\usepackage{physics}
\usepackage{siunitx}
\usepackage{hyperref}

\begin{document}

\section*{Lösung zu H2.1 Kugelwellen}

Wir betrachten die zeitabhängige Schrödinger-Gleichung für ein freies Teilchen der Masse \(m\) in drei Dimensionen,
\[
i\hbar\,\frac{\partial}{\partial t}\Psi(\vb*{x},t)
=
-\frac{\hbar^2}{2m}\,\nabla^2\Psi(\vb*{x},t)\,.
\]
Im Folgenden zeigen wir, dass sich in \( \vb*{x}\neq \vb*{0} \) spezielle Lösungen in Form von Kugelwellen
\[
\Psi(\vb*{x},t)
= A\,\frac{1}{r}\,\exp\bigl(i\,(k\,r - \omega\,t)\bigr),
\qquad
r = \lvert \vb*{x}\rvert,
\quad
\omega = \frac{\hbar k^2}{2m},
\]
finden lassen.

\subsection*{(a) Nachweis der Kugelwellen-Lösung}

\paragraph{Ansatz.}
Wir setzen
\[
\Psi(\vb*{x},t)
= A\,\frac{e^{i(kr - \omega t)}}{r},
\]
mit Konstanten \(A\), \(k\) und \(\omega\).

\paragraph{Laplacian in Kugelkoordinaten.}
Für radialsymmetrische Funktionen \(f(r)\) gilt in Kugelkoordinaten
\[
\nabla^2 f(r)
= \frac{1}{r^2}\,\frac{\dd}{\dd r}
\Bigl(r^2\,\frac{\dd f}{\dd r}\Bigr).
\]
Man zeigt (durch Einsetzen von \(f(r) = e^{ikr}/r\)), dass
\[
\nabla^2\Bigl(\frac{e^{ikr}}{r}\Bigr)
= -k^2\,\frac{e^{ikr}}{r}.
\]
Damit ergibt sich auf der rechten Seite der Schrödinger-Gleichung
\[
-\frac{\hbar^2}{2m}\,\nabla^2\Psi
= -\frac{\hbar^2}{2m}\,
\bigl(-k^2\,\frac{e^{i(kr-\omega t)}}{r}\bigr)\,A
= \frac{\hbar^2 k^2}{2m}\,\Psi.
\]
Die linke Seite liefert
\[
i\hbar\,\frac{\partial\Psi}{\partial t}
= i\hbar\,\bigl(-i\omega\bigr)\,
A\,\frac{e^{i(kr-\omega t)}}{r}
= \hbar\omega\,\Psi.
\]
Beide Seiten stimmen überein, wenn
\[
\hbar\omega = \frac{\hbar^2 k^2}{2m}
\;\;\Longrightarrow\;\;
\omega = \frac{\hbar k^2}{2m},
\]
was zu zeigen war.

\subsection*{(b) Wahrscheinlichkeitsstromdichte und Kontinuitätsgleichung}

\paragraph{Definition des Wahrscheinlichkeitsstroms.}
Für eine Wellenfunktion \(\Psi\) ist der Strom
\[
\vb*{j}(\vb*{x},t)
= \frac{\hbar}{2mi}\,\bigl(\Psi^*\,\nabla\Psi - \Psi\,\nabla\Psi^*\bigr).
\]
Da \(\Psi\) nur von \(r\) abhängt und \(\nabla \Psi = \hat{\vb*{r}}\,\frac{\dd\Psi}{\dd r}\),
ergibt sich
\[
\frac{\dd}{\dd r}\Bigl(\frac{e^{i(kr-\omega t)}}{r}\Bigr)
= \frac{e^{i(kr-\omega t)}}{r^2}\,(ikr - 1),
\]
also
\[
\nabla\Psi
= \hat{\vb*{r}}\,
A\,\frac{e^{i(kr-\omega t)}}{r^2}\,(ikr - 1).
\]
Einsetzen liefert nach etwas Rechnung
\[
j_r(r)
= \hat{\vb*{r}}\cdot\vb*{j}
= \frac{\hbar k}{m}\,\frac{\lvert A\rvert^2}{r^2}.
\]
Alle Tangentialkomponenten verschwinden, da keine Winkelabhängigkeit vorliegt.

\paragraph{Kontinuitätsgleichung.}
Die Kontinuitätsgleichung fordert
\[
\frac{\partial \rho}{\partial t} + \nabla\!\cdot\!\vb*{j} = 0,
\qquad
\rho = \lvert \Psi\rvert^2 = \frac{\lvert A\rvert^2}{r^2}.
\]
Offensichtlich ist \(\rho\) zeitunabhängig, also \(\partial_t\rho=0\). Weiter gilt
\[
\nabla\!\cdot\!\vb*{j}
= \frac{1}{r^2}\,\frac{\dd}{\dd r}\bigl(r^2 j_r(r)\bigr)
= \frac{1}{r^2}\,\frac{\dd}{\dd r}\Bigl(r^2\,\frac{\hbar k}{m}\,\frac{\lvert A\rvert^2}{r^2}\Bigr)
= 0.
\]
Somit ist die Kontinuitätsgleichung erfüllt.

\subsection*{(c) Wahrscheinlichkeitsfluss durch geschlossene Oberflächen}

Wir betrachten
\[
\Phi_S
= \oint_S \vb*{j}\cdot \dd\vb*{S}
= \oint_S j_r(r)\,r^2\,\dd\Omega
= \frac{\hbar k}{m}\,\lvert A\rvert^2
\;\oint_S \dd\Omega.
\]

\begin{itemize}
  \item[(i)] Wenn \(S\) den Ursprung \((r=0)\) einschließt, deckt die Fläche alle Richtungen ab, also
  \(\displaystyle\oint_S \dd\Omega = 4\pi\). Damit
  \[
    \Phi_S = \frac{4\pi\hbar k}{m}\,\lvert A\rvert^2
    \quad(\text{für }0\in V_S).
  \]
  \item[(ii)] Schließt \(S\) den Ursprung nicht ein, liegt im inneren Volumen kein Quellen-/Senkenpunkt,
  und da \(\nabla\!\cdot\!\vb*{j}=0\) für \(r\neq0\), ergibt sich per Gaußscher Integralsatz
  \(\Phi_S=0\).
\end{itemize}

Physikalisch bedeutet dies: Die Kugelwelle transportiert \emph{Outward}-Fluss, der aus
einem punktförmigen Ursprung \(r=0\) „emittiert“ wird; Ringsherum hat man einen konstanten
stromaufbauenden Fluss.

\end{document}
