\documentclass{article}
\usepackage{amsmath, amssymb}
\begin{document}

\section*{Aufgabe H1.2: Gaußsche Integrale (3 Punkte)}

Zeigen Sie, dass für die Gaußschen Integrale die folgenden Identitäten gelten:
\begin{enumerate}
  \item[(a)]
    \[
    \int_{-\infty}^{+\infty} e^{-\alpha x^2}\,dx = \sqrt{\frac{\pi}{\alpha}}, \quad \text{Re}(\alpha) > 0.
    \]
  \item[(b)]
    \[
    \int_{-\infty}^{+\infty} e^{-\alpha x^2+\beta x}\,dx = \sqrt{\frac{\pi}{\alpha}}\, e^{\frac{\beta^2}{4\alpha}}, \quad \text{Re}(\alpha) > 0.
    \]
\end{enumerate}

\textbf{Hinweis:} Für den Fall (a) berechnen Sie zuerst das Quadrat des Integrals in Polarkoordinaten.

\bigskip

\section*{Lösung}

\subsection*{(a) Berechnung des Integrals \(\displaystyle I(\alpha)=\int_{-\infty}^{+\infty} e^{-\alpha x^2}\,dx\)}

Zunächst betrachten wir das Quadrat des Integrals:
\[
I(\alpha)^2 = \left( \int_{-\infty}^{+\infty} e^{-\alpha x^2}\,dx \right)^2 = \int_{-\infty}^{+\infty}\int_{-\infty}^{+\infty} e^{-\alpha(x^2+y^2)}\,dx\,dy.
\]
Wechseln wir nun in Polarkoordinaten:
\[
x = r\cos\theta,\quad y = r\sin\theta,\quad \text{und}\quad dx\,dy = r\,dr\,d\theta.
\]
Damit erhalten wir:
\[
I(\alpha)^2 = \int_0^{2\pi} \int_0^\infty e^{-\alpha r^2} \, r\,dr\, d\theta.
\]
Das \(\theta\)-Integral lässt sich direkt berechnen:
\[
\int_0^{2\pi} d\theta = 2\pi.
\]
Für das \(r\)-Integral führen wir die Substitution \(u=\alpha r^2\) durch, woraus \(du=2\alpha r\,dr\) folgt, also:
\[
r\,dr = \frac{du}{2\alpha}.
\]
Somit:
\[
\int_0^\infty r\, e^{-\alpha r^2}\,dr = \frac{1}{2\alpha} \int_0^\infty e^{-u}\,du = \frac{1}{2\alpha}.
\]
Es ergibt sich:
\[
I(\alpha)^2 = 2\pi \cdot \frac{1}{2\alpha} = \frac{\pi}{\alpha}.
\]
Daraus folgt nach Wurzelziehung (und unter Beachtung, dass \(I(\alpha)>0\)):
\[
I(\alpha) = \sqrt{\frac{\pi}{\alpha}}.
\]

\subsection*{(b) Berechnung des Integrals \(\displaystyle J(\alpha,\beta)=\int_{-\infty}^{+\infty} e^{-\alpha x^2+\beta x}\,dx\)}

Wir schreiben zunächst den Exponenten um:
\[
-\alpha x^2+\beta x = -\alpha \left(x^2-\frac{\beta}{\alpha}x\right).
\]
Um das Quadrat zu vervollständigen, beachten wir:
\[
x^2-\frac{\beta}{\alpha}x = \left(x-\frac{\beta}{2\alpha}\right)^2 - \frac{\beta^2}{4\alpha^2}.
\]
Somit kann der Exponent geschrieben werden als:
\[
-\alpha x^2+\beta x = -\alpha \left(x-\frac{\beta}{2\alpha}\right)^2 + \frac{\beta^2}{4\alpha}.
\]
Einsetzen in das Integral liefert:
\[
J(\alpha,\beta)= \int_{-\infty}^{+\infty} e^{-\alpha \left(x-\frac{\beta}{2\alpha}\right)^2 + \frac{\beta^2}{4\alpha}}\,dx 
= e^{\frac{\beta^2}{4\alpha}} \int_{-\infty}^{+\infty} e^{-\alpha \left(x-\frac{\beta}{2\alpha}\right)^2}\,dx.
\]
Mit der Substitution \( u = x-\frac{\beta}{2\alpha} \) (wobei \(du=dx\)) ändert sich das Integral zu:
\[
\int_{-\infty}^{+\infty} e^{-\alpha u^2}\,du,
\]
welches nach (a) den Wert
\[
\sqrt{\frac{\pi}{\alpha}}
\]
hat. Somit erhalten wir:
\[
J(\alpha,\beta)= e^{\frac{\beta^2}{4\alpha}} \sqrt{\frac{\pi}{\alpha}}.
\]

\bigskip

\section*{Zusammenfassung}

\[
\boxed{
\begin{aligned}
\int_{-\infty}^{+\infty} e^{-\alpha x^2}\,dx &= \sqrt{\frac{\pi}{\alpha}}, \quad \text{Re}(\alpha) > 0,\\[1mm]
\int_{-\infty}^{+\infty} e^{-\alpha x^2+\beta x}\,dx &= \sqrt{\frac{\pi}{\alpha}}\, e^{\frac{\beta^2}{4\alpha}}, \quad \text{Re}(\alpha) > 0.
\end{aligned}
}
\]

\end{document}
