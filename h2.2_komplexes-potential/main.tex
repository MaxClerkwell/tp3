\documentclass[a4paper,12pt]{article}
\usepackage[T1]{fontenc}
\usepackage[utf8]{inputenc}
\usepackage[ngerman]{babel}
\usepackage{amsmath,amssymb}
\usepackage{physics}
\usepackage{csquotes}
\usepackage{hyperref}
\sloppy

\begin{document}

\section*{Lösung zu H2.2: Komplexes Potential}

Wir betrachten die zeitabhängige Schrödinger-Gleichung mit einem komplexen Potential
\[
V(\mathbf{x}) = v(\mathbf{x}) + i\,w(\mathbf{x}),
\]
wobei $v(\mathbf{x})$ und $w(\mathbf{x})$ reelle Funktionen sind.

\subsection*{(a) Herleitung der modifizierten Kontinuitätsgleichung}

Die Schrödinger-Gleichung und ihre komplex Konjugierte lauten
\begin{align}
  i\hbar\,\frac{\partial}{\partial t}\Psi(\mathbf{x},t) &= \bigl[-\tfrac{\hbar^2}{2m}\nabla^2 + v(\mathbf{x}) + i\,w(\mathbf{x})\bigr]\,\Psi(\mathbf{x},t), \\
  -i\hbar\,\frac{\partial}{\partial t}\Psi^*(\mathbf{x},t) &= \bigl[-\tfrac{\hbar^2}{2m}\nabla^2 + v(\mathbf{x}) - i\,w(\mathbf{x})\bigr]\,\Psi^*(\mathbf{x},t).
\end{align}
Wir definieren die Wahrscheinlichkeitsdichte
\[
\rho(\mathbf{x},t) = \Psi^*(\mathbf{x},t)\,\Psi(\mathbf{x},t)
\]
und den Wahrscheinlichkeitsstrom
\[
\mathbf{j}(\mathbf{x},t) = \tfrac{\hbar}{2mi}\bigl(\Psi^*\nabla\Psi - (\nabla\Psi^*)\,\Psi\bigr).
\]
Die zeitliche Ableitung von $\rho$ ergibt
\begin{align*}
  \frac{\partial \rho}{\partial t}
  &= \Psi^*\,\frac{\partial \Psi}{\partial t} + \frac{\partial \Psi^*}{\partial t}\,\Psi \\
  &= \Psi^*\,\frac{1}{i\hbar}\bigl[-\tfrac{\hbar^2}{2m}\nabla^2 + v + i\,w\bigr]\,\Psi
     - \Psi\,\frac{1}{i\hbar}\bigl[-\tfrac{\hbar^2}{2m}\nabla^2 + v - i\,w\bigr]\,\Psi^*.
\end{align*}
Die Terme des reellen Potentials und die kinetischen Terme führen auf den Divergenzbeitrag $-\nabla\cdot\mathbf{j}$, übrig bleibt
\[
\frac{\partial \rho}{\partial t} + \nabla\cdot\mathbf{j} = -\frac{2}{\hbar}\,w(\mathbf{x})\,\rho(\mathbf{x},t).
\]
Zum Vergleich: Für $V=v(\mathbf{x})$ echt wäre RHS~$=0$ und man erhielte
\[
\frac{\partial \rho}{\partial t} + \nabla\cdot\mathbf{j} = 0.
\]

\subsection*{(b) Erhaltung der Gesamtwahrscheinlichkeit}

Die Norm (Gesamtwahrscheinlichkeit) ist
\[
P(t) = \int_{\mathbb{R}^3} \rho(\mathbf{x},t)\,\mathrm{d}^3x.
\]
Deren Zeitableitung gibt mit der modifizierten Kontinuitätsgleichung
\begin{align*}
  \frac{\mathrm{d}P}{\mathrm{d}t}
  &= \int_{\mathbb{R}^3} \frac{\partial \rho}{\partial t}\,\mathrm{d}^3x
   = -\int_{\mathbb{R}^3} \nabla\cdot\mathbf{j}\,\mathrm{d}^3x
     - \tfrac{2}{\hbar}\int_{\mathbb{R}^3}w(\mathbf{x})\,\rho(\mathbf{x},t)\,\mathrm{d}^3x.
\end{align*}
Der erste Term verschwindet, wenn $\mathbf{j}\to0$ im Unendlichen, daher
\[
\boxed{\frac{\mathrm{d}P}{\mathrm{d}t} = -\frac{2}{\hbar} \int_{\mathbb{R}^3} w(\mathbf{x})\,\rho(\mathbf{x},t)\,\mathrm{d}^3x.}
\]

\begin{itemize}
  \item Für $w(\mathbf{x})>0$ nimmt $P(t)$ ab: partielle Absorption (\enquote{Senke}).
  \item Für $w(\mathbf{x})<0$ wächst $P(t)$: Verstärkung (\enquote{Quelle}).
  \item Im Allgemeinen ist $P(t)$ nicht erhalten, im Gegensatz zum Fall eines reellen Potentials.
\end{itemize}

\textbf{Interpretation:} Der imagi­näre Anteil des Potentials wirkt wie Quelle bzw. Senke von Teilchen; Modelle offener Quantensysteme verwenden solche Terme für dissipative Prozesse.

\end{document}

