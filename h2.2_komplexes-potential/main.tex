\documentclass[a4paper,12pt]{article}
\usepackage[utf8]{inputenc}
\usepackage{amsmath}
\usepackage{amssymb}
\usepackage{physics}
\usepackage{hyperref}

\begin{document}

\section*{Lösung zu H2.2: Komplexes Potential}

Wir betrachten die zeitabhängige Schrödinger-Gleichung im Einklang mit einem komplexen Potential
\[
V(\mathbf{x}) = v(\mathbf{x}) + i\,w(\mathbf{x}),
\]
wobei \(v(\mathbf{x})\) und \(w(\mathbf{x})\) reelle Funktionen sind.

\subsection*{(a) Herleitung der modifizierten Kontinuitätsgleichung}

Die Schrödinger-Gleichung und ihre komplex konjugierte lauten
\begin{align}
  i\hbar\,\frac{\partial}{\partial t}\Psi(\mathbf{x},t) &= \Bigl[-\frac{\hbar^2}{2m}\nabla^2 + v(\mathbf{x}) + i\,w(\mathbf{x})\Bigr]\Psi(\mathbf{x},t), \label{schroedinger}\\
  -i\hbar\,\frac{\partial}{\partial t}\Psi^*(\mathbf{x},t) &= \Bigl[-\frac{\hbar^2}{2m}\nabla^2 + v(\mathbf{x}) - i\,w(\mathbf{x})\Bigr]\Psi^*(\mathbf{x},t).
\end{align}
Wir definieren die Wahrscheinlichkeitsdichte
\[
\rho(\mathbf{x},t) = \Psi^*(\mathbf{x},t)\,\Psi(\mathbf{x},t)
\]
und den Wahrscheinlichkeitsstrom
\[
\mathbf{j}(\mathbf{x},t)
= \frac{\hbar}{2mi}\Bigl(\Psi^*\nabla\Psi - (\nabla\Psi^*)\,\Psi\Bigr).
\]
Leiten wir nun \(\rho\) nach der Zeit ab:
\begin{align*}
  \frac{\partial \rho}{\partial t}
  &= \Psi^*\,\frac{\partial \Psi}{\partial t}
     + \frac{\partial \Psi^*}{\partial t}\,\Psi \\
  &= \Psi^*\,
     \frac{1}{i\hbar}\Bigl[-\tfrac{\hbar^2}{2m}\nabla^2 + v + i\,w\Bigr]\Psi
     \;-\;\Psi\,
     \frac{1}{i\hbar}\Bigl[-\tfrac{\hbar^2}{2m}\nabla^2 + v - i\,w\Bigr]\Psi^*.
\end{align*}
Die Terme mit dem reellen Anteil \(v\) heben sich weg, ebenso die kinetischen Terme geben nach partieller Integration genau den Divergenzbeitrag \(-\nabla\cdot\mathbf{j}\). Übrig bleibt jedoch
\[
\frac{\partial \rho}{\partial t}
+ \nabla\cdot\mathbf{j}
= -\frac{2}{\hbar}\;w(\mathbf{x})\;\rho(\mathbf{x},t).
\]
Diese Gleichung ist die \emph{modifizierte Kontinuitätsgleichung} für ein komplexes Potential. 
Zum Vergleich: Für ein rein reelles Potential \(V=v\) wäre der rechte Seiten-Ausdruck Null, und es gilt
\[
\frac{\partial \rho}{\partial t} + \nabla\cdot\mathbf{j} = 0,
\]
also die gewöhnliche Kontinuitätsgleichung.

\subsection*{(b) Erhaltung der Gesamtwahrscheinlichkeit}

Wir betrachten die Gesamtwahrscheinlichkeit (bzw. Norm)
\[
P(t) \;=\; \int_{\mathbb{R}^3} \rho(\mathbf{x},t)\,\mathrm{d}^3x.
\]
Differenzieren wir nach der Zeit und nutzen die modifizierte Kontinuitätsgleichung,
\begin{align*}
  \frac{\mathrm{d}P}{\mathrm{d}t}
  &= \int_{\mathbb{R}^3} \frac{\partial \rho}{\partial t}\,\mathrm{d}^3x
   = -\int_{\mathbb{R}^3} \nabla\cdot\mathbf{j}\,\mathrm{d}^3x
     - \frac{2}{\hbar}\int_{\mathbb{R}^3} w(\mathbf{x})\,\rho(\mathbf{x},t)\,\mathrm{d}^3x.
\end{align*}
Der erste Integralterm verschwindet, sofern \(\mathbf{j}\) am Unendlichen ausreichend schnell gegen Null geht. Damit ergibt sich
\[
\boxed{
\frac{\mathrm{d}P}{\mathrm{d}t}
= - \frac{2}{\hbar}\int_{\mathbb{R}^3} w(\mathbf{x})\,\rho(\mathbf{x},t)\,\mathrm{d}^3x.
}
\]
\begin{itemize}
  \item Ist \(w(\mathbf{x}) > 0\) in einem Gebiet, so nimmt \(P(t)\) ab: Das System \emph{verliert} Wahrscheinlichkeit, was man als partielle \emph{Absorption} oder Austritt von Teilchen interpretieren kann.
  \item Ist \(w(\mathbf{x}) < 0\), so wächst \(P(t)\): Man spricht von einer \emph{Verstärkung} oder Quelle von Teilchen.
  \item Insgesamt ist \(P(t)\) im Allgemeinen \emph{nicht} erhalten, im Gegensatz zum Fall eines reellen Potentials.
\end{itemize}

\medskip

\textbf{Interpretation:} Ein imaginärer Anteil des Potentials kann als effektiver „Quell‐“ bzw. „Senken“-Term für Teilchen betrachtet werden. In der Theorie offener Quanten­systeme wird dies oft als Modell für dissipative Prozesse (z.\,B. Teilchenverlust durch Reaktionen, Absorption in einem Bildgebungsprozess) verwendet.

\end{document}
