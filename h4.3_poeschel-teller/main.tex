\documentclass[a4paper,11pt]{article}
\usepackage[utf8]{inputenc}
\usepackage{amsmath,amssymb}
\usepackage{hyperref}
% Definition der sech-Funktion
\DeclareMathOperator{\sech}{sech}
\begin{document}

\section*{Lösung zu Aufgabe H4.3: Pöschl–Teller–Potential}

Wir betrachten die stationäre Schrödinger­gleichung in einer Dimension
\[
-\frac{\hbar^2}{2m}\,\frac{\mathrm{d}^2\Psi(x)}{\mathrm{d}x^2}
+V(x)\,\Psi(x)=E\,\Psi(x),
\]
mit dem Pöschl–Teller–Potential
\[
V(x)=-\frac{\lambda(\lambda+1)\,\hbar^2}{2m}\,\sech^2(x),
\quad \lambda\in\mathbb{N}_0.
\]

\subsection*{(a) Substitution und Herleitung der Differentialgleichung}

\begin{enumerate}
  \item Schrödinger­gleichung umgeformt:
  \[
    -\frac{\hbar^2}{2m}\Psi''(x)
    -\frac{\lambda(\lambda+1)\,\hbar^2}{2m}\sech^2(x)\,\Psi(x)
    =E\,\Psi(x),
  \]
  Multiplizieren mit $-2m/\hbar^2$:
  \[
    \Psi''(x)+\Bigl(\tfrac{2mE}{\hbar^2}+\lambda(\lambda+1)\sech^2(x)\Bigr)\,\Psi(x)=0.
  \]
  Für gebundene Zustände ($E<0$) setzt man
  \[
    \mu^2:=-\frac{2mE}{\hbar^2}, 
    \quad E=-\frac{\hbar^2\mu^2}{2m},
  \]
  und erhält
  \[
    \Psi''(x)-\bigl(\mu^2-\lambda(\lambda+1)\sech^2(x)\bigr)\,\Psi(x)=0.
  \]

  \item Mit $\sech^2(x)=1-\tanh^2(x)$ wählen wir die Substitution
  \[
    u=\tanh(x), 
    \quad \frac{\mathrm{d}u}{\mathrm{d}x}=\sech^2(x), 
    \quad \frac{\mathrm{d}^2u}{\mathrm{d}x^2}=-2\tanh(x)\sech^2(x).
  \]
  Setze $\Phi(u)=\Psi(x)$. Nach Anwendung der Kettenregel folgt
  \[
    (1-u^2)\,\Phi''(u)-2u\,\Phi'(u)
    +\Bigl[\lambda(\lambda+1)-\tfrac{\mu^2}{1-u^2}\Bigr]\,\Phi(u)=0.
  \]
  Daraus erkennt man, dass die Energie über
  \[
    E=-\frac{\hbar^2\mu^2}{2m}
  \]
  vom Parameter $\mu$ abhängt.
\end{enumerate}

\subsection*{(b) Lösung und Energiespektrum}

\begin{enumerate}
  \item Die allgemeine Lösung der assoziierten Differentialgleichung ist
  \[
    \Phi(u)=A\,P_\lambda^{\mu}(u)+B\,Q_\lambda^{\mu}(u),
  \]
  wobei \(P\) und \(Q\) die assoziierten Legendre­funktionen sind. Für gebundene, normalisierbare Zustände auf \((-1,1)\) muss \(B=0\) und \(\mu\) ganzzahlig sein:
  \[
    \mu=\lambda-n,\quad n=0,1,2,\dots,\lambda.
  \]

  \item Die normierten Wellenfunktionen lauten
  \[
    \Psi_n(x)
    =N_n\,P_\lambda^{\,\lambda-n}\!\bigl(\tanh(x)\bigr),
    \quad n=0,1,\dots,\lambda,
  \]
  mit Normierungskonstanten \(N_n\).

  \item Das diskrete Energiespektrum ist
  \[
    E_n
    =-\frac{\hbar^2}{2m}\,(\lambda-n)^2,
    \quad n=0,1,\dots,\lambda.
  \]
  Es gibt also genau \(\lambda+1\) gebundene Zustände mit
  \(\;E_0< E_1<\cdots< E_{\lambda}<0\).
\end{enumerate}

\end{document}
